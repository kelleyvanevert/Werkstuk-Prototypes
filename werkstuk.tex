\documentclass
  [11pt,
   paper=a4,
   cleardouble=plain,
   chapterprefix=true,
   parskip=half,
   draft=true,
  ]
  {scrbook}

\usepackage[dutch]{babel}
\usepackage[T1]{fontenc}
\usepackage[utf8]{inputenc}

\usepackage{amsmath}
\usepackage{amssymb}
\usepackage{marvosym}

\usepackage{pxfonts}

\usepackage{tikz}

\usepackage{listings}
\lstdefinelanguage{proto}
  {keywords={var, obj, clones, fun, returns, print, skip, if, else, while},
   emph={true,false,this,proto},
   comment=[l]{\#}}
\lstdefinelanguage{class}[]{proto}
  {morekeywords={extends},
   deletekeywords={clones},
   moreemph={super},
   deleteemph={proto}}
\lstnewenvironment{program}{}{}
\lstMakeShortInline @
\lstset
  {language=proto,
   basicstyle=\small\ttfamily,
   emphstyle=\scshape,
   numbers=left,
   numberstyle=\tiny,
   numbersep=2em,
   extendedchars=true}

\usepackage{syntax}
\setlength{\grammarindent}{8em}

\frenchspacing% Cruciaal!
\raggedbottom

\begin{document}

\title{Een natuurlijke semantiek voor prototype gebaseerde talen}
\author{Kelley van Evert \& Tim Steenvoorden}
\maketitle

\frontmatter

\tableofcontents

\mainmatter

\chapter{Syntaxis}

\section{Grammatica}

Ik heb er een officiële \textsc{BNF}-grammatica van gemaakt en gezet met het @syntax@-pakket.
Enkele opmerkingen:
\begin{description}
  \item[Statements]
    \begin{itemize}
      \item Compositie gebeurt met een nieuwe regel,
        dit gegeven we aan met \Righttorque.
        Blokken geven we aan met behulp van inspringen,
        zodat we geen @end@ nodig hebben of haakjes.
      \item In plaats van @skip@ kunnen we dus ook \Righttorque gebruiken.
      \item @if@ en @while@ zijn standaard.
      \item @print@ om uitvoer van programma's te kunnen testen.
      \item @var@ is een \emph{declaratie}, en @=@ een \emph{toekenning}.
      \item @fun f() ...@ is een \emph{definitie}, @f()@ een \emph{aanroep}.
      \item Om objecten te kunnen klonen heb ik @obj A clones B@ verzonnen.
        Dit lijkt op het declareren van variabelen met @var@
        (maar dan @obj@ wegens objecten)
        en geeft een relatie tussen de objecten weer met @clones@.
        Dit kunnen we later eventueel uitbreiden naar klassen door
        @clones@ te vervangen door @extends@.
    \end{itemize}
  \item[Parameters \& Argumenten]
    Spreken redelijk voor zich. Achter de komma staat een spatie, aangegeven met \textvisiblespace.
  \item[Identifiers \& Slots]
    \begin{itemize}
      \item Onze ideeën over \emph{locaties} en \emph{paden}
        hernoemd naar \emph{slots}. Een slot is een plek waaronder
        je iets kunt opslaan (variabele, functie, object).
        Deze term wordt vaak gebruikt bij talen als Lua en IO.
      \item Identifiers bestaan gewoon uit kleine letters en hoofdletters.
        (Weet jij nog een goede manier om dit aan te geven zonder het hele
        alfabet op te noemen?)
      \item Het lijkt mij een goed idee om de namen van objecten te beginnen
        met een hoofdletter en variabelen en functies met een kleine.
    \end{itemize}
  \item[Expressies \& Operatoren]
    Alleen gehele getallen. Operatoren zijn standaard aritmetisch.
  \item[Tests \& Booleans]
    Speciale syntactische categorie zodat we ze alleen kunnen gebruiken binnen
    @if@ en @while@. Bevatten standaard logische operatoren.
    We kunnen de @&@ en @|@ ook vervangen door @&&@ en @||@ of @/\@ en @\/@,
    wat denk jij?
    Relaties zijn equivalentie en ordeningen op gehele getallen.
    Ik heb @true@ en @false@ woordelijk toegevoegd, overbodig of goed idee?
\end{description}

\begin{grammar}
<Statement>  ::= <Statement> \Righttorque <Statement>
            \alt \Righttorque
            \alt @if@ <Test> <Statement> @else@ <Statement>
            \alt @while@ <Test> <Statement>
            \alt @print@ <Expression>
            \alt @var@ <Slot> 
            \alt <Slot> "=" <Expression>
            \alt @fun@ <Slot> "(" <Parameters> ")" <Statement>
            \alt <Slot> "(" <Arguments> ")"
            \alt @obj@ <Slot> @clones@ <Slot>

<Parameters> ::= <Identifier> ",\ " <Parameters> | <Identifier>

<Arguments>  ::= <Expression> ",\ " <Arguments>  | <Expression>

<Slot>       ::= <Identifier> "." <Slot> | <Identifier>

<Identifier> ::= "a" | "b" | "c" | \dots

<Expression> ::= <Number>
            \alt <Slot>
            \alt <Expression> <Operator> <Expression>

<Operator>   ::= "+" | "-" | "*" | "/" | "\%"

<Number>     ::= "0" | "1" | "2" | "3" | \dots

<Test>       ::= <Boolean>
            \alt <Test> "\&" <Test>
            \alt <Test> "|"  <Test>
            \alt "~" <Test>
            \alt <Expression> <Relation> <Expression>

<Relation>   ::= "==" | "!=" | "<" | "<=" | ">" | ">="

<Boolean>    ::= @true@ | @false@
\end{grammar}

\section{Een programma over deuren}

Programma's kunnen we eenvoudig zetten met het @listings@-pakket. We kunnen @\begin{program}@ en @\end{program}@ gebruiken om programma's tussen te schrijven. Daarnaast heb ik het apenstaartje zo ingesteld dat alles tussen twee apenstaartjes in ``code''-modus worden gezet, inclusief het opmaken van sleutelwoorden!

\begin{program}
obj Door clones Object       // Maak een nieuwe deur

fun Door.open()              // Een functie om de deur te openen
    print 42

obj LockedDoor clones Door   // Maak een nieuwe gesloten deur...
var LockedDoor.isLocked
LockedDoor.isLocked = 1      // ...die dicht is

fun LockedDoor.unlock(code)
    if code == this.code     // Als de code correct is...
        isLocked = 0         // ...openen we de deur

fun LockedDoor.open()        // Herdefinieer om isLocked te gebruiken
    if isLocked == 1
        print 0
    else
        proto.open()         // Expliciete aanroep naar prototype

obj Safe clones LockedDoor   // Maak een kluis...
var Safe.code
Safe.code = 1234             // ...en ken een code toe

Safe.unlock(4321)
Safe.open()                  // Geeft 0, de deur is gesloten
Safe.unlock(1234)
Safe.open()                  // Geeft 42, we de juiste code hadden
\end{program}

\section{Een programma over personen}

\begin{program}
obj Person clones Object

var Person.total
Person.total = 0

fun Person.initialize()      // Wordt aangeroepen bij iedere kloon
    Person.total = Person.total + 1

obj Me clones Person
var Me.age
Me.age = 37

obj You clones Person
var You.age
You.age = 42

print Person.total           // Geeft 2
\end{program}

\section{Een programma over scope}

\begin{program}
var x
x = 1
fun g()
    print x
    x = 2
fun f()
    var x
    x = 3
    g()
f()
print x
\end{program}

% vim: spell spl=nl

%\chapter{Troep}

\section{Regels}

\subsection{While}

@$\<$while not (x = 1) do y = y * x; x = x - 1$, s_{3,2}\> \to s_{6,1}$@

$\<\WHILE x \neq 1 \DO y = y \times x;~ x = x - 1, s_{3,2}\> \to s_{6,1}$

$\<\mathtt{while}(x \neq 1, y = y \times x;~ x = x - 1), s_{3,2}\> \to s_{6,1}$

$\<\FUN counter(n) \RETURNS next \IS (\FUN next() \RETURNS n \IS n = n + 1), s\>$

$\<\FUN counter(n) \RETURNS next \IS \FUN next() \RETURNS n \IS n = n + 1 \END \END, s\>$

$\<counter := \mathtt{block}(n, \mathtt{block}(n = n + 1)), s\>$

$\<\VAR x\COMP x=1\COMP
   \FUN g() \IS (\PRINT x\COMP x=2)\COMP
   \FUN f() \IS (\VAR x\COMP x=3\COMP g())\COMP
   f()\COMP
   \PRINT x,
s\> \to s'$


$\<\VAR x\COMP x=1\COMP
   \FUN g() \IS \PRINT x\COMP x=2 \END\COMP
   \FUN f() \IS \VAR x\COMP x=3\COMP g() \END\COMP
   f()\COMP
   \PRINT x,
s\> \to s'$

$\<x:=1;~
   g:=\mathtt{method}(x \mathtt{~print};~ x=2);~
   f:=\mathtt{method}(x:=3;~ g());~
   f();~
   x \mathtt{~print}, 
s\> \to s'$

$\<x:=1;
   g:=\mathtt{block}(x \mathtt{~println}; x=2);
   f:=\mathtt{block}(x:=3; g \mathtt{~call});
   f \mathtt{~call};
   x \mathtt{~println}, 
s\> \to s'$

\subsection{Fun}

\<@fun $f$($y$) {$x$ = $y$}@, $s$, $\epsilon$\> $\to (s',\epsilon)$

@$\<$fun $f$($y$) {$x$ = $y$}$, s, \epsilon\> \to (s',\epsilon)$@

\<@fun f(y) {x = y}@, $s$, $\epsilon$\> $\to (s',\epsilon)$

$\<\FUN f(y) \IS x = y, s, \epsilon\> \to (s', \epsilon)$

---

$\< \mathbf{fun}\;f(y)\;\;\{x=y\}, s, \epsilon \> \to s'$

$\<\texttt{\FUN\ f(y) \{x = y\}}, s, \epsilon\> \to s'$

$\<\syn{fun f(y) {x = y}}, s, \epsilon\> \to s'$

%$\<@fun@ f@(@y@)@ @{@ x @=@ y@}@, s, \epsilon\> \to s'$

$\<\mathtt{fun} f(y) \{x = y\}, s, \epsilon\> \to s'$

\<@fun@ $f$@(@$y$@)@ @{@ $x$@=@$y$ @}@, $s$, $\epsilon$\> $\to s'$

--- ---

\subsection{If}

\<@if $T$ $S_1$ else $S_2$@, $s$, $\epsilon$\> $\to (s',\epsilon)$

@$\<$if $T$ $S_1$ else $S_2, s, \epsilon\> \to (s',\epsilon)$@

\<@if b T else F@, $s$, $\epsilon$\> $\to (s',\epsilon)$

---

$\<\syn{if T S1 else S2}, s\> \to s'$

$\<\IF\; T\; S_1\; \ELSE\; S_2, s\> \to s'$

%$\langle @if@ T S_1 @else@ S_2, s \rangle \to s'$

$\langle$@if T S1 else S2@$\rangle$

@$\langle$ if $T$ $S_1$ else $S_2$ $\rangle$@

\<@if@ $T$ $S_1$ @else@ $S_2$, $s$\> $\to s'$

%$\< @if@\; T\; S_1\; @else@\; S_2, s \> \to s'$

$\< \IF T S_1 \ELSE S_2 \> \to s'$

%\tuple{@if@ $T$ $S_1$ @else@ $S_2$}$\to s'$

\section{Bomen}

\begin{tikzpicture}
   [edge from parent path={
      ($(\tikzchildnode.west) - (0,0.5\tikzleveldistance)$)
      -- ($(\tikzchildnode.east) - (0,0.5\tikzleveldistance)$)
      ($(\tikzparentnode.west) + (0,0.5\tikzleveldistance)$)
      -- ($(\tikzparentnode.east) + (0,0.5\tikzleveldistance)$)
    },
    grow'=up,
    level/.style={sibling distance=20em/#1},
    level distance=4ex]
    \node {$\<\WHILE x \neq 1 \DO y = y \times x\COMP x = x - 1, s_{3,2}\> \to s_{6,1}$}
    child { node {$\<y = y \times x\COMP x = x - 1, s_{3,2}\> \to s_{6,1}$}
      child { node {$\<y = y \times x, s_{3,2}\> \to s_{6,2}$} }
    child { node {$\<x = x - 1, s_{6,2}\> \to s_{6,1}$} } }
  child { node {$\<\WHILE x \neq 1 \DO y = y \times x\COMP x = x - 1, s_{6,1}\> \to s_{6,1}$} };
\end{tikzpicture}

% vim: ft=context


\backmatter

\end{document}

% vim: spell spl=nl
