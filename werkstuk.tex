\documentclass[11pt,oneside,parskip=half]{scrbook}
\usepackage[a4paper]{geometry}

\usepackage[dutch]{babel}
\usepackage[T1]{fontenc}
\usepackage[utf8]{inputenc}

\usepackage{amsmath}
\usepackage{amssymb}

%\usepackage{pxfonts}
%\usepackage{millennial}
\usepackage{fouriernc}
\usepackage{tgschola}
\usepackage{tgadventor}
%\usepackage{newcent}
%\usepackage[utopia]{mathdesign}

\usepackage{tabularx}% tables: equal width columns
\usepackage{multirow}% tables: rowspan
\usepackage{booktabs}% tables: \toprule, \bottomrule, ...
\usepackage{graphicx}
\usepackage{xcolor}
\usepackage{xargs}
\usepackage{enumitem}
\usepackage{csquotes}
\usepackage{ulem}
\usepackage{float}
\usepackage{ifthen}
\usepackage[font=small,it]{caption}
\usepackage{environ}
\usepackage{setspace}
\doublespacing

\usepackage{bussproofs}
  \def\extraVskip{4pt} % Iets meer verticale ruimte
  % De volgende commands gebruik ik om op een simpele manier (1) de toegepaste regel aan de rechterkant van de deductieregel te laten zien, en (2) een automatische nummering aan de linkerkant te laten zien.
  % Normaal gesproken gebruik je \RightLabel en \LeftLabel, maar nu gebruik ik dus: \Label (enkel een label) en \CountedLabel (label en automatische nummering)
  % Terugzetten op 1 kan met "\setcounter{theLabel}{1}", bijvoorbeeld bij een nieuwe boom
  \newcounter{theLabel}
  \setcounter{theLabel}{1}
  \newcommand{\Label}[1]{
    \RightLabel{\small #1}
  }
  \newcommand{\CountedLabel}[1]{
    \RightLabel{\small #1}
    \LeftLabel{\scriptsize \arabic{theLabel}}
    \addtocounter{theLabel}{1}
  }
  % Op een of andere manier moet je in je deductiebomen, de states gaan aangeven. Ik doe dit met doosjes, waarin net genoeg staat dat je iets intuitiever kunt zien wat welke state is, en wat er allemaal in zit.
  \newcommand{\stateBox}[1]{%
    \boxed{\mathit{\vphantom{f}#1}}
  }
  \newcommand{\emptyStateBox}{\stateBox{\phantom{f}}}

\usepackage{tikz}
\usetikzlibrary{calc}
\tikzstyle{every picture}=[thick]

\usepackage{listings}
\lstdefinelanguage{proto}
  {keywords={local, object, clones, function, returns, is,
             print, skip,
             if, then, else, while, do},
   emph={true, false, this, proto},
   %literate={==}{{$=$}}1 {/=}{{$\neq$}}1 {<=}{{$\leq$}}1 {>=}{{$\geq$}}1
   %         {+}{{$+$}}1 {-}{{$-$}}1 {*}{{$\times$}}1 {/}{{$/$}}1
   %         {&}{{$\wedge$}}1 {|}{{$\vee$}}1 {~}{{$\lnot$}}1,
   comment=[l]{--}}
\lstMakeShortInline @
\lstset
  {language=proto,
   basicstyle=\small\ttfamily,
   commentstyle=\color{gray},
   keywordstyle=\bfseries,
   emphstyle=\scshape,
   numbers=left,
   numberstyle=\tiny,
   numbersep=1em,
   xleftmargin=3em,
   mathescape=true,
   extendedchars=true}
\renewcommand{\lstlistingname}{Voorbeeld}

\usepackage[nounderscore]{syntax}
\setlength{\grammarindent}{8em}
\renewcommand{\syntleft}{\normalfont\itshape}
\renewcommand{\syntright}{}

\newcommand\newkeyword[2]
  {\newcommand{#1}
     {\raisebox{.3pt}{\text{\scalebox{.95}{\small\ttfamily\bfseries #2}}}}}
\newcommand\newconstant[2]{\newkeyword{#1}{#2}}
\newcommand{\newnonterminal}[2]
  {\newcommand{#1}
    {\textit{#2}}}
\newcommand{\newsemanticset}[2]
  {\newcommand{#1}
    {\ensuremath{\mathbb{#2}}}}

\newcommand{\syn}[1]
  {\ensuremath{\mathit{#1}}}

\newcommand{\COMP}{;\;}
\newkeyword{\LOCAL}{local\;}
\newkeyword{\VAR}{\LOCAL}
\newkeyword{\OBJ}{obj\;}
\newkeyword{\CLONES}{\;clones\;}
\newkeyword{\OBJECT}{\;object}
\newkeyword{\FUN}{function\;}
\newkeyword{\RETURNS}{\;returns\;}
\newkeyword{\IS}{\;is\;}
\newkeyword{\PRINT}{print\;}
\newkeyword{\SKIP}{skip}
\newkeyword{\IF}{if\;}
\newkeyword{\THEN}{\;then\;}
\newkeyword{\ELSE}{\;else\;}
\newkeyword{\WHILE}{while\;}
\newkeyword{\DO}{\;do\;}
\newkeyword{\END}{\;end}

\newconstant{\TRUE}{true}
\newconstant{\FALSE}{false}
\newconstant{\SELF}{self}
\newconstant{\THIS}{this}
\newconstant{\PROTO}{proto}

\newconstant{\AND}{and}
\newconstant{\OR}{or}
\newconstant{\NOT}{not}

\newnonterminal{\Stm}{Stm}
\newnonterminal{\Id}{Id}
\newnonterminal{\Ids}{Ids}
\newnonterminal{\MaybeIds}{MaybeIds}
\newnonterminal{\Slot}{Slot}
\newnonterminal{\Expr}{Expression}
\newnonterminal{\Exprs}{Expressions}
\newnonterminal{\ExprsMaybe}{MaybeExpressions}
\newnonterminal{\B}{BooleanExpression}
\newnonterminal{\Num}{Num}

\newsemanticset{\sO}{O}
\newsemanticset{\sS}{S}
\newsemanticset{\sB}{B}
\newsemanticset{\sM}{M}
\newsemanticset{\sL}{L}
\newsemanticset{\sF}{F}
\newsemanticset{\sV}{V}
\newsemanticset{\sN}{N}

\def\GrammarDef{::=\;&}
\def\GrammarOr{\mid\;&}
\newcommand{\GrammarOpt}[1]{[#1]\;}

\def\IN{\quad}

\def\I{\textit}

\newcommand{\<}
  {\ensuremath{\langle}}
\renewcommand{\>}
  {\ensuremath{\rangle}}

\frenchspacing% Cruciaal!
\raggedbottom% Handig

\floatstyle{plain}
\newfloat{code}{H}{code.list.aux}[chapter]
\floatname{code}{Code fragment}

\newcounter{CodeFragment}[chapter]
\renewcommand{\theCodeFragment}{\thechapter.\arabic{CodeFragment}}
\NewEnviron{CodeFragment}[1][]{%
  \refstepcounter{CodeFragment}%
  %\setcounter{CodeFragmentLineNo}{0}%
  \vspace{-1pc}%
  \begin{code}
    \ifthenelse{\equal{#1}{}}{\caption{}}{\caption{#1}}%
	  \begin{equation*}%
		  \begin{array}{l@{\hspace*{.02\textwidth}}|@{\hspace*{.02\textwidth}}l}%
		  \hspace*{.35\textwidth} & \hspace*{.55\textwidth} \\[-1pc]%
		  \BODY%
		  \end{array}%
	  \end{equation*}%
	  \vspace{-1pc}%
	\end{code}%
  \vspace{-1pc}%
}

\newcounter{CodeFragmentLineNo}[CodeFragment]
\newcommand{\Line}[2]{%
  \refstepcounter{CodeFragmentLineNo}%
  \text{\color{gray} \scriptsize \arabic{CodeFragmentLineNo}}\;\;%
  #1 &%
  \ifthenelse{\equal{#2}{}}{}{\textit{\small- #2}}%
  \\
}

% todo if second param empty? then don't display scalebox etc..
\newcommand{\AxiomOrInference}[2]{%
  \renewcommand{\arraystretch}{1.2}%
  \begin{minipage}{\textwidth}% prevent page break
    #1% = \begin{prooftree} ... \end{prooftree}
    \vspace*{-.5pc}%
    \hspace*{.45\textwidth}% ideally: .5\textwidth - 1/2 "desda\quad"
    \scalebox{.85}{%
      #2%
    }%
  \end{minipage}%
}

\NewEnviron{NSAxiom}[1]{%
  \renewcommand{\arraystretch}{1.2}%
  \begin{minipage}{\textwidth}% prevent page break
  \BODY%
  \end{minipage}%
}
\NewEnviron{NSConditions}{%
  \vspace*{-.5pc}%
  \hspace*{.45\textwidth}% ideally: .5\textwidth - 1/2 "desda\quad"
  \scalebox{.85}{%
    \begin{tabular}{@{}r@{\quad}l@{}}%
      \textbf{desda} \BODY%
    \end{tabular}%
  }%
}
\newcommand{\Cond}[1]{%
  & #1 \\
}

\newcommand{\FiniteFunctions}[2]{\ensuremath{{#1}^{\!\!\raisebox{4pt}{\rotatebox{70}{$\Lsh$}}\!\!#2}}}

\newcommand{\surrAngle}[1]{\ensuremath{\big<\,#1\,\big>}}
\newcommand{\surr}[1]{\ensuremath{\big[\,#1\,\big]}}
\newcommand{\Surr}[1]{\ensuremath{\big[\!\big[\,#1\,\big]\!\big]}}

% configuration: " <S, m, \sigma, \tau> "
%\newcommand{\Config}[2]{\left<#1, #2\right>}%  -- normaal
%\newcommand{\Config}[2]{\surr{#1}_{#2}}%       -- [ ]m,s,t
\newcommand{\Config}[2]{\surrAngle{#1}_{#2}}%  -- < >m,s,t

\def\BOT{\rotatebox{90}{$\Vdash$}}
\let\nil\BOT

\def\DEF{\buildrel{\text{def}}\over{=}}
\def\Proto{\sqsubset^p}
\def\Outer{\sqsubset^s}
\def\ProtoEq{\sqsubseteq^p}
\def\OuterEq{\sqsubseteq^s}
\def\Attr{\textit{attr}}
\def\SDef{\textit{def}}
\newcommand{\ScopeID}[1]{\ensuremath{(1, #1)}}
\newcommand{\ObjectID}[1]{\ensuremath{(2, #1)}}

\def\KelleyColor{\color[HTML]{417438}}
\NewEnviron{Kelley}{{\KelleyColor \BODY}}
\newcommand{\KelleySaysNo}[1]{{\KelleyColor \sout{#1}}}
\newcommand{\KelleySays}[1]{{\KelleyColor \uwave{#1}}}
\NewEnviron{KelleyWouldRemove}{%
  \begin{minipage}{.1\textwidth}\KelleyColor \ttfamily [rm]\end{minipage}%
  \begin{minipage}{.9\textwidth}\BODY\end{minipage}%
  \vspace*{3pt}%
}
\NewEnviron{KelleyWouldAdd}{%
  \begin{minipage}{.1\textwidth}\KelleyColor \ttfamily [add]\end{minipage}%
  \begin{minipage}{.9\textwidth}\KelleyColor \BODY\end{minipage}%
  \vspace*{3pt}%
}

\title{Een natuurlijke semantiek voor prototype oververing en lexicaal bereik}
\author{Kelley van Evert \& Tim Steenvoorden}

\begin{document}

\maketitle

\frontmatter

\tableofcontents

\mainmatter


\chapter{Inleiding}

% Wat
In dit werkstuk presenteren we een natuurlijke semantiek die wij ontworpen hebben om de concepten \emph{lexicaal bereik} en \emph{prototype overerving} in \emph{object-geörienteerde} talen te karakteriseren. Daartoe hebben we een minimale taal ontworpen die geïnspireerd is door de bestaande programmeertalen IO en JavaScript.

% Waarom
Lexicaal bereik (ook wel \emph{static scoping} genaamd) en prototype overerving zijn mooie fenomenen. Ze zijn ook de fundamenten van ``The World's Most Misunderstood Programming Language'': JavaScript. Maar lexicaal bereik ligt men eigenlijk heel natuurlijk: zo redeneren wiskundigen al meer dan honderd jaar met formules waarin variabelen lexicaal bereik hebben. En prototype overerving is slechts een elegant en simpel alternatief op klassieke overerving, wanneer het gaat om object-geörienteerd programmeren.

% Doel
Het doel van dit werkstuk is daarom een formele betekenis te geven aan deze concepten, maar dan wel zó dat de interpretatie van de formele uitspraken zo natuurlijk mogelijk en conceptueel verantwoord is. De bedoeling is dus dat je de gewoon Nederlandse interpretatie van een willekeurig axioma of deductieregel tegen zou kunnen komen in een college programmeren:

\begin{multicols}{2}
  \small
  \raggedcolumns
  \setlength{\columnseprule}{.5pt}
  \scalebox{0.86}{\begin{minipage}{.25\textwidth}
  \begin{NSAxiom}{object}
  \begin{prooftree}
    \AxiomC{$
      \Config{i \OBJECT}{m, \sigma, \tau}
      \longrightarrow
      m''
    $}
  \end{prooftree}
  \begin{NSConditions}
    \Cond{$ \textsc{Find}_m(\sigma, i) = \sigma_\text{def} $}
    \Cond{$ m' = m \surr{ \sigma_\text{def} \mapsto \big(b_{m(\sigma')}[i\mapsto \omega], p_{m(\sigma')}\big) } $}
    \Cond{$ m'' = m' \surr{ \omega \mapsto \big(\varnothing, \BOT\big) } $}
  \end{NSConditions}
\end{NSAxiom}
  \end{minipage}}
  
  \columnbreak
  
  \textit{``Zoals jullie weten, moeten we bij statische scope eerst de definie van de variabele zoeken in de huidige en daarna omliggende scopes. Daarna kan een nieuw object worden gemaakt en in de heap gezet, en een \emph{pointer} naar dit object wordt vervolgens in de variabele gestopt\dots''}
\end{multicols}

% Opzet
Na het bespreken van een aantal notationele keuzes en terminologie, presenteren we eerst de minimale taal, vervolgens het semantische model en tenslotte de natuurlijke semantiek die de twee voorgaande aan elkaar koppelt. In de case study die erop volgt gaan we uitvoeriger in op \dots


\chapter{Notatie en terminologie}

\section{Functies}

In dit werkstuk identificeren we een functie met zijn grafiek, d.w.z.~een functie $f : X \to Y$ is gelijk aan de verzameling paren $(x, y) \in X \times Y$ waarvoor geldt: $f(x) = y$.

\section{Partiële functies}

Vrijwel alle functies die we in dit werkstuk behandelen zijn partiële functies. Wanneer een partiële functie $f$ niet gedefiniëerd is op een zeker punt $x$, schrijven we $f(x) = \bot$. Wanneer het omgekeerde het geval is, schrijven we $f(x) \neq \bot$.

Voor een willekeurige term $\phi = \dots f(x)\dots$, waarbij $f$ een partiële functie is die niet gedefiniëerd is op punt $x$, geldt dat $\phi = \bot$. Op deze manier hoeven we niet iets omslachtigs te schrijven als: ``als $f(x) = \bot$, dan $z = \bot$; anders als $f(x) \neq \bot$, dan $z = \phi$''.

\section{Tupels}

We zullen meermaals in ons werkstuk gebruik maken van willekeurig grote, maar altijd eindige, ``lijsten'' van elementen uit een zekere verzameling: \emph{tupels}. Deze tupels worden gerepresenteerd door eindige (partiële) functies $t : \sN \to X$, als X de verzameling element in kwestie is, waaraan nog een paar extra voorwaarden worden gesteld. De verzameling van alle tupels op een zekere verzameling $X$, genoteerd $X_{\langle\rangle}$, is als volgt gedefiniëerd:

$$ X_{\langle\rangle} \DEF \big\{\, t : \sN \to X \mid \exists_{N \in \sN} \big[ \forall_{n < N}[t(n) \neq \bot] \land \forall_{n\ge N}[t(n) = \bot] \big] \,\big\} $$

We schrijven $\langle\rangle$, maar ook wel $\varnothing$, voor de lege tupel (deze is natuurlijk hetzelfde voor elke waarden verzameling $X$).

We schrijven $\langle x_0, x_1, \dots, x_{N-1}\rangle$ voor de tupel $t$ waarvoor geldt: $\forall_{n < N}[t(n) = x_n]$, en: $\forall_{n \ge N}[t(n) = \bot]$.

Als $t$ een tupel is $\in X_{\langle\rangle}$, en $x$ een element van $X$, dan schrijven we $t:x$ voor de tupel $t' = t[N \mapsto x]$, waarbij $N = \min\{n\in\sN \mid t(n) = \bot\}$.

\section{Beschouwing semantisch model}

We definieren in dit werkstuk een natuurlijke semantiek, d.w.z.~een ?-ste orde logica, met axioma's en deductieregels, en een bijbehorende structuur waarin deze zich afspeelt.

Deze structuur, die we ook wel het \emph{semantisch model} zullen noemen, heeft onderstaand opgesomde elementen. Deze worden verderop precies gedefinieerd, onderstaande opsomming geeft slechts een algemeen beeld.

\begin{description}
	\item[$\mathbb{M}$]\hfill\\ De verzameling mogelijke \emph{geheugens}, welke ook wel als \emph{eindtoestanden} worden geinterpreteerd.
	\item[$(\mathit{Stm} \times \mathbb{M} \times \mathbb{L} \times \mathbb{L})$]\hfill\\ De verzameling \emph{toestanden}, ook wel \emph{configuraties}.
	\item[$(\longrightarrow)$]\hfill\\ Een tweeplaatsig predikaat welke als eerste argument een element uit de verzameling van toestanden neemt, en als tweede argument een element uit de verzameling van eindtoestanden $(\mathbb{M}\dots)$. De uitspraak $(S, m, \sigma, \tau) \longrightarrow m'$ moet worden geinterpreteerd worden als:
	\begin{quote} ``Het programma $S$, met geheugen $m$, in scope $\sigma$ en met als $\mathbf{this}$ object $\tau$, resulteert in eindtoestand $m'$, mits $S$ \emph{correct} is''. \end{quote}
\end{description}

\section{Notationele conventies}

Terwille van elegantie houden we een aantal gebruikelijke notationele conventies aan:

\begin{enumerate}
	\item Voor elke twee willekeurige tweestemmige predikaten $\mathsf{S}$ en $\mathsf{T}$ (mogelijk ook $=$), en drie willekeurige elementen $a$, $b$ en $c$, definieren we de afkorting: $$a \operatorname{\mathsf{S}} b \operatorname{\mathsf{T}} c \buildrel{\mathrm{def}}\over{=} a \operatorname{\mathsf{S}} b \land b \operatorname{\mathsf{T}} c$$ in het geval dat deze bewering correct getypeerd is.
	\item Op eenzelfde manier definieren we ook de volgende afkorting: $$ \{a \in A \mid \phi \} \buildrel{\mathrm{def}}\over{=} \{a \mid a \in A \mid \phi\}$$
\end{enumerate}

[...]


\chapter{Taal en syntaxis}

In dit hoofdstuk presenteren we de taal presenteren waarvoor we een natuurlijk semantiek construeren. De taal maakt gebruikt van prototype overerving en lexicaal bereik. Eerst beschouwen we een aantal voorbeeldprogramma's, om zo informeel de te formaliseren taal te karakteriseren. Daarna geven we een rigoreuze definitie met behulp van een \BNF\ grammatica.

De structuur van de productieregels van deze grammatica worden in latere hoofdstukken gebruikt om axioma's en deductieregels op te stellen. Daarmee heeft de grammatica in zekere zin een dubbele functie.

Het is belangrijk om te vermelden dat het ons hierbij niet gaat om een ``goed uitziende'' taal te maken, maar enkel om er de essentiële onderdelen in te verwerken die nodig zijn voor ons doeleinde: het formaliseren van lexicaal bereik en prototype overerving met een natuurlijke semantiek. Om diezelfde reden moet de syntaxis van de taal voornamelijk worden beschouwd als wat een mogelijke representatie van een \emph{abstract syntax tree} van een ``echte'' programmeertaal zou kunnen zijn.

\section{Voorbeeldprogramma's}
\label{sec:voorbeelden}

Elk voorbeeldprogramma en zijn toelichtingen worden als volg gepresenteerd:

\newCodeFragment[exa:template][Het eerste voorbeeldprogramma]
\codeFragmentCaption
\codeLines{
  \codeLine{\VAR \I{f}}[$\I{f}$ moet eerst worden gedefiniëerd]
  \codeLine{\I{f} = \FUN(i) \RETURNS \I{n}}
  \codeLine{\IN \VAR \I{n}}
  \codeLine{\IN \I{n} = 2 \times (\I{i} + 5)}
  \codeLine{}[$\I{x}$ bestaat niet in deze scope]
  \codeLine{\VAR \I{x}}[\I{x} heeft nog geen waarde, maar is wel gedefiniëerd]
  \codeLine{\I{x} = \I{f}(42)}[\I{x} heeft nu de waarde 94]
}

De toelichtingen moeten als informeel commentaar worden beschouwd, waarmee we aan proberen te geven hoe het programma zich gedraagt. Vaak zijn het uitspraken over de toestand waarin het programma zich bevindt, direct na de linker regel te hebben ``uitgevoerd''.

\subsection{Basis}

\subsubsection{Declaratie van variabelen}

Een variabele moet altijd eerst worden gedeclareerd, alvorens er een waarde aan kan worden toegekend of het op andere manieren kan worden gebruikt. Een programma waarin variabelen worden gereferenceerd die nooit zijn gedefiniëerd is niet valide. In code fragment \ref{exa:declaration} zie je hoe declaratie plaatsvindt.

\newCodeFragment[exa:declaration][Declaratie van variabelen]
\codeFragmentCaption
\codeLines{
  \codeLine{}[\I{x} bestaat (nog) niet]
  \codeLine{\VAR \I{x}}[\I{x} heeft nog geen waarde, maar is wel gedefiniëerd]
  \codeLine{\I{x} = 5}[\I{x} bevat nu de waarde 5]
}

Het concept van declaratie is juist in deze taal heel belangrijk, gezien het lexicaal bereik van variabelen. Wat lexicaal bereik precies inhoudt wordt weldra behandeld.

\subsubsection{Types}

Variabelen kunnen na declaratie waarden aannemen. Onze taal bevat waarden van drie types (natuurlijk getal, functie, object), en het onderscheid tussen deze types wordt op \emph{dynamisch} niveau gemaakt i.p.v~op syntactisch niveau. Dat houdt in dat een willekeurige variabele elke willekeurige waarde kan aannemen, van elk willekeurig type. Ook kan het in zijn levensduur waarden van meerder types bevatten. Code fragment \ref{exa:types} geeft dit idee weer.

\newCodeFragment[exa:types][Waarden van verschillende types]
\codeFragmentCaption
\codeLines{
  \codeLine{\VAR \I{x}}[declaratie zonder type indicatie]
  \codeLine{\I{x} = 5}[type ``natuurlijk getal'']
  \codeLine{\I{x} = \FUN()\{\dots\}}[type ``functie'']
  \codeLine{\I{x} \OBJECT}[type ``object'', deze rare initiälisatie heeft een reden]
}

\subsection{Lexicaal bereik}

De \emph{scope} (ook wel \emph{bereik}) van een variabele in een programma, is het deel van het programma waarin zij kan worden gebruikt. En zijn verschillende manieren om dit bereik te definiëren, een daarvan is \emph{lexicaal bereik} (ook wel \emph{lexical} of \emph{static scoping}) er één.

% [Tim:] ik kan dit niet echt mooi verwoorden ...?
Gegeven een bepaalde definitie van dit bereik, is de vraag: Als ik een variabele naam tegenkom, over welke variabele heb ik het dan? Code fragmenten \ref{exa:zoek} en \ref{exa:zoek2} illustreren deze vraag. Met \emph{lexicaal bereik} wordt deze dus ruwweg in de ``lexicale omgeving'' van de referentie gezocht.

\newCodeFragment[exa:zoek][Zoek de definitie van variabele \I{x}]
\codeFragmentCaption
\codeLines{
  \codeLine{\VAR \I{x}}[dit is de gezochte definitie van \I{x}]
  \codeLine{\I{x} = 42}
  \codeLine{}
  \codeLine{\VAR \I{f}}
  \codeLine{\I{f} = \FUN()}
  \codeLine{\IN \dots \I{x} \dots}[waar is deze \I{x} gedefiniëerd?]
}

\newCodeFragment[exa:zoek2][Zoek de definitie van variabele \I{x}]
\codeFragmentCaption
\codeLines{
  \codeLine{\VAR \I{x}}
  \codeLine{\I{x} = 42}
  \codeLine{}
  \codeLine{\VAR \I{f}}
  \codeLine{\I{f} = \FUN()}
  \codeLine{\IN \VAR \I{x}}[dit is de gezochte definitie van \I{x}]
  \codeLine{\IN \I{x} = 43}
  \codeLine{\IN \dots \I{x} \dots}[waar is deze \I{x} gedefiniëerd?]
}

Als je de code zó indenteert zoals in bovenstaande twee code fragmenten (\ref{exa:zoek} en \ref{exa:zoek2}): bij elke functie definitie wordt een regel geïndenteerd, dan komt de indentatie ongeveer overeen met de boomstructuur van de scopes. Het moge duidelijk zijn dat als men ``naar buiten toe'' zoekt naar de definities van variabelen, een zekere boomstructuur van toepassing is.

\subsection{Object oriëntatie en prototype overerving}

De taal is object georienteerd.

\newCodeFragment
\codeLines{
  \codeLine{\VAR \I{o}}
  \codeLine{\I{o} \OBJECT}
  \codeLine{}                            [$\I{o}.\I{f}$ is niet gedefinieerd]
  \codeLine{\I{o}.\I{f} = \FUN()\,\{\;\SKIP\;\}} [toekenning waarde aan object attribuut]
  \codeLine{}                            [$\I{o}.\I{f}$ is wel gedefinieerd]
  \codeLine{\I{o}.\I{n} = 5}
}

Van de drie typen, zijn getallen en functies \emph{primitief}, en objecten \emph{niet primitief}. Primitieve waarde worden zelf gekopieerd (\emph{by-value}), maar van niet-primitieve waarden worden \emph{referenties} gekopieerd (\emph{by-reference}).

\newCodeFragment
\codeLines{
  \codeLine{\VAR \I{x};\; \I{x} = 6}
  \codeLine{\VAR \I{y};\; \I{y} = \I{x}}             [$\I{x} = 6$ en $\I{y} = 6$]
  \codeLine{\I{y} = 7}                       [$\I{x} = 6$ en $\I{y} = 7$]
  \codeLine{}
  \codeLine{\VAR \I{p};\; \I{p}.\I{n} = 6}
  \codeLine{\VAR \I{q};\; \I{q} = \I{p}}             [$\I{p}$ en $\I{q}$ verwijzen nu naar hetzelfde object]
  \codeLine{}                            [$\I{p}.\I{n} = 6$ en $\I{q}.\I{n} = 6$]
  \codeLine{\I{q}.\I{n} = 7}                     [$\I{p}.\I{n} = 7$ en $\I{q}.\I{n} = 7$]
}

\subsection{Lexical scope}

Als in een zekere scope een variabele wordt gereferenceerd (nog) niet is gedefinieerd, wordt in omliggende scopes ``gezocht'' naar een definitie van deze variabele.

\newCodeFragment[exa:lexical]

\codeLines{
  \codeLine{\VAR \I{x}}
  \codeLine{\VAR \I{f};\; \I{f} = \FUN(i)}
  \codeLine{\IN \I{x} = \I{i} + 5}
  \codeLine{}
  \codeLine{\I{f}(5)}[$\I{x} = 10$]
}

..maar wanneer deze wel in de huidige scope bestaat, worden omliggende scopes ``met rust gelaten''.

\newCodeFragment

\codeLines{
  \codeLine{\VAR \I{x}}
  \codeLine{\VAR \I{f}}
  \codeLine{\I{f} = \FUN(i)}
  \codeLine{\IN \VAR \I{x}}
  \codeLine{\IN \I{x} = \I{i} + 5}
  \codeLine{}
  \codeLine{\I{f}(5)}[$\I{x}$ heeft nog geen waarde]
}

Telkens wanneer een functie wordt aangeroepen, wordt een \emph{nieuwe scope} aangemaakt voor lokale variabelen. Variabelen van deze nieuwe scope kunnen later nog gereferenceerd worden, doordat bijvoorbeeld de functie een lokale functie teruggeeft.

\newCodeFragment

\codeLines{
  \codeLine{\VAR \I{f}}
  \codeLine{\I{f} = \FUN(n) \RETURNS \I{g}}
  \codeLine{\IN \VAR \I{g}}
  \codeLine{\IN \I{g} = \FUN() \RETURNS \I{n}}
  \codeLine{\IN \IN \I{n} = \I{n} + 1}
  \codeLine{}
  \codeLine{\VAR \I{c}}
  \codeLine{\I{c} = \I{f}(5)}[$\I{c}() \rightarrow 6, 7, 8, \dots$]
}

\begin{lstlisting}[caption=Een countervoorbeeld,label=exa:counter]
local f
f = function($n$) returns g
    local g
    g = function() returns n
        n = n + 1

local c
c = f(5)
c()                          # $\rightarrow 6, 7, 8, \dots$
\end{lstlisting}

\fbox{maar dan wat beter geschreven, etc...}

\subsection{Prototype overerving}

Prototype overerving is een variant van object-geörienteerd programmeren. De kern van object-geörienteerd programmeren is het concept van een \emph{object}, dat ertoe dient een verschijnsel uit de werkelijkheid na te bootsen (een reëel object, een patroon, een abstract idee). Het doel is om meer te kunnen programmeren op een conceptueel niveau. Daarmee wordt bijvoorbeeld zowel creatie als onderhoud van de code makkelijker.

Veel objecten zullen natuurlijk gelijke eigenschappen vertonen, of dezelfde structuur hebben. Verder wilt men concepten als specificering en generalisering toepassen op objecten. Deze problemen kunnen op meerdere manieren worden aangepakt. De bekendste variant is \emph{klasse gebaseerde} object-oriëntatie (ook wel \emph{klassieke object-oriëntatie}) en richt zich op het concept van een \emph{klasse}. Objecten van een bepaalde klasse vertonen de structuur en gedrag van die klasse en heten \emph{instanties}. Van specificering is sprake als een klasse eigenschappen van een andere klasse \emph{overerft}. Klassieke object-oriëntatie vind men in talen als Java en C\#.

Een andere aanpak met hetzelfde doel is \emph{prototype gebaseerde} object-oriëntatie. Daarbij wordt geen scheiding gemaakt tussen de concepten klasse, die structuur en gedrag specificeert, en instantie, die enkel deze eigenschappen vertoont. In plaats daarvan wordt gewerkt met een prototype structuur, waarbij elk object naar een bepaald \emph{prototype}-object refereert. Nu zijn objecten zelf de dragers van structuur en gedrag.
%Deze methode kan als flexibeler worden gezien, maar ook als een wat minder reëel beeld van de werkelijkheid worden opgevat.

Technisch gezien werkt prototype overerving als volgt. Van elk object is een prototype bekend, of het heeft geen prototype. Wanneer men een attribuut opvraagt van een zeker object, kan de op te leveren waarde procedureel als volgt worden opgevat:

\begin{enumerate}
  \item Bekijk of het attribuut gedefiniëerd is in het object zelf. In dat geval weten we de waarde en leveren deze op.
  \item Anders zoeken we het attribuut op in het prototype van het object. Ook dan weten we de waarde en leveren deze op.
  \item Wanneer ook het prototype het attribuut niet bevat, herhalen we de zoektocht voor alle volgende prototypen totdat we het attribuut hebben gevonden.
\end{enumerate}

Het grote verschil tussen object-gebaseerde talen en prototype-gebaseerde talen is dus dat de tweede geen onderscheid maakt tussen klassen en instanties. Een prototype heeft beide functies. Neem bijvoorbeeld het object \I{Deur}:

\newCodeFragment

\codeLines{
  \codeLine{\VAR \I{Deur}}
  \codeLine{\I{Deur} \OBJECT}
}

We kunnen \I{Deur} direct als instantie gebruiken door een attribuut te zetten:

\codeLines{
  \codeLine{\I{Deur}.\I{open} = 1}
}

Een \I{Deur} is standaard open. We kunnen \I{Deur} ook als een prototype gebruiken. In prototype-gebaseerde talen heet dit \emph{klonen}:

\codeLines{
  \codeLine{\VAR \I{GeslotenDeur}}
  \codeLine{\I{GeslotenDeur} \OBJECT}
	\codeLine{\I{GeslotenDeur} \CLONES \I{Deur}}
}

\I{GeslotenDeur} heeft dan alle attributen van \I{Deur}:

\codeLines{
  \codeLine{\I{GeslotenDeur}.\I{open}}[waarde $\to$ 1]
}

Maar een \I{GeslotenDeur} moet natuurlijk gesloten zijn. We zetten zijn attribuut \I{open} op @0@:

\codeLines{
	\codeLine{\I{GeslotenDeur}.\I{open} = 0}
}

Een gewone \I{Deur} is nog steeds open:

\codeLines{
  \codeLine{\I{Deur}.\I{open}}[waarde $\to$ 1]
}

Attributen worden dus per object bewaard. Door \I{open} op @0@ te zetten in \I{GeslotenDeur} verandert er niks in \I{Deur}.

We kunnen net zoveel klonen maken van een object als we willen en net zo diep klonen als we willen. Neem een \I{GlazenDeur}, dit is natuurlijk ook een \I{Deur}, maar wel doorzichtig:

\codeLines{
  \codeLine{\VAR \I{GlazenDeur}}
  \codeLine{\I{GlazenDeur} \OBJECT}
  \codeLine{\I{GlazenDeur} \CLONES \I{Deur}}
  \codeLine{\I{GlazenDeur}.\I{doorzichtig} = 1}
}

Een gewone \I{Deur} heeft het attribuut \I{doorzichtig} niet, en dus een \I{GeslotenDeur} ook niet:

\codeLines{
  \codeLine{\I{GeslotenDeur}.\I{doorzichtig}}[fout!]
}

Maar we kunnen besluiten dat deuren standaard niet doorzichtig zijn:

\codeLines{
  \codeLine{\I{Deur}.\I{doorzichtig} = 0}
}

Zodat ook onze \I{GeslotenDeur} niet doorzichtig is:

\codeLines{
  \codeLine{\I{GeslotenDeur}.\I{doorzichtig}}[waarde $\to$ 0]
}

Maar er geld nog steeds:

\codeLines{
  \codeLine{\I{GlazenDeur}.\I{doorzichtig}}[waarde $\to$ 1]
}

We zien dat we met prototypes een zeer flexibele methode hebben om object-geörienteerd te programmeren. Het is niet nodig om de compiler of parser van te voren uit te leggen dat objecten aan bepaalde ``blauwdrukken'' moeten voldoen. We creëren objecten ``on-the-fly'', alsmede hun attributen en relaties. Deze methode komt terug in talen als JavaScript, IO en Self.

Natuurlijk is het ook mogelijk om \emph{methoden} te definiëren. Dit zijn functie attributen gekoppeld aan een specifiek object. Stel dat we een \I{GeslotenDeur} graag open willen maken. We definiëren:

\codeLines{
  \codeLine{\I{GeslotenDeur}.\I{ontsluit} = \FUN (\I{poging})}
  \codeLine{\IN \IF (\I{poging} = \THIS.\I{code}) \THEN}
  \codeLine{\IN \IN \THIS.\I{open} = 1}
  \codeLine{\IN \ELSE}
  \codeLine{\IN \IN \THIS.\I{open} = 0}
}

\I{this} is hier een expliciete verwijzing naar het huidige object. Op dit moment kunnen we \I{ontsluit} nog niet aanroepen op \I{GeslotenDeur}:

\codeLines{
  \codeLine{\I{GeslotenDeur}.\I{ontsluit}(1234)}[fout!]
}

Het attribuut \I{code} is immers niet gedefinieerd in \I{GeslotenDeur} noch in zijn prototype \I{Deur}.

We kunnen natuurlijk een \I{code} toekennen aan \I{GeslotenDeur}, maar laten we een specifieke \I{GeslotenDeur} maken met een \I{code}:

\codeLines{
  \codeLine{\VAR \I{Kluis}}
  \codeLine{\I{Kluis} \OBJECT}
  \codeLine{\I{Kluis} \CLONES \I{GeslotenDeur}}
  \codeLine{\I{Kluis}.\I{code} = 4321}
}

Wanneer we de methode \I{ontsluit} aanroepen is deze niet gedefinieerd in \I{Kluis}, maar wel in zijn prototype \I{GeslotenDeur}. Die wordt dan uitgevoerd. Een belangrijke observatie is dat \I{ontsluit} wel wordt aangeroepen op \I{Kluis}. Dat betekent dat \I{this} verwijst naar \I{Kluis} en niet \I{GeslotenDeur}. Het attribuut \I{code} wordt dan wel gevonden:

\codeLines{
  \codeLine{\I{Kluis}.\I{ontsluit}(1234)}
  \codeLine{\I{Kluis}.\I{open}}[waarde $\to$ 0]
}

Helaas was dat de verkeerde code, we proberen het nog een keer:

\codeLines{
  \codeLine{\I{Kluis}.\I{ontsluit}(1234)}
  \codeLine{\I{Kluis}.\I{open}}[waarde $\to$ 1]
}

%TODO: Meerdere overerving niet mogelijk?

\section{Grammatica}

Nu volgt een formele definitie van de syntaxis van de taal, aan de hand van een \BNF\ grammatica. Nummers zijn als volgt gedefiniëerd:
\begin{align*}
  \Num \GrammarDef (0 \mid 1 \mid 2 \mid 3 \mid 4 \mid 5 \mid 6 \mid 7 \mid 8 \mid 9)^+
\end{align*}
Eigenlijk gebruiken we geen strikte \BNF, in deze specifieke gevallen, maar een hele simpele variant, zoals \textsc{E-BNF}, waarbij de ook simpele reguliere expressies toelaten.% T:, die ook reguliere expressies toelaat.
Bovenstaand voorbeeld maakt dit duidelijk. Voorbeelden van elementen uit $\Num$ zijn ``0'', ``1'', ``235783'' en ``0003''. Voorbeelden van elementen die niet in $\Num$ zitten zijn ``'', ``-6'', ``4.2''.

\emph{Identifiers}, die gebruikt worden als variabel-namen,% T: namen voor variabelen en attributen
zijn op eenzelfde manier als volgt gedefiniëerd:
\begin{align*}
  \Id \GrammarDef (a \mid b \mid c \mid d \mid \dots)^+
\end{align*}
Hierbij moet je je natuurlijk% T: moet men zich
voorstellen dat alle letters uit het alfabet in de grammaticaregel staan op de voordehandliggende% T: voor de hand liggende
manier.

Het is soms ook nodig om meerdere komma-gescheiden namen te gebruiken, of een mogelijk lege lijst, zoals bij functie definities. Vandaar de volgende twee productieregels:
\begin{align*}
  \Ids \GrammarDef \Id \mid \Ids, \Id \\
  \MaybeIds \GrammarDef \varepsilon \mid \Ids
\end{align*}
Een \emph{slot}% T: pad, ook binnen de grammatica veranderen
is een opeenvolging door punten (``.'') gescheiden identifiers,% T: opeenvolging van identifiers gescheiden door punten
en wordt gebruikt om ook naar attributen van objecten te kunnen referenceren:% T: refereren
\begin{align*}
  \Slot \GrammarDef \Id \mid \Id . \Slot
\end{align*}
\emph{Expressies}, die ofwel primitieve waarden (getallen en functies), ofwel objecten kunnen weergeven, en \emph{boolse expressies}, die gebruikt worden voor loops en conditionele executie, zijn als volgt gedefiniëerd:% T: definiëren we als volgt
\begin{align*}
  \Expr \GrammarDef \Num \mid \Slot \mid \varnothing \mid \Expr\; (+ \mid - \mid \times \mid /\,)\; \Expr \\
  \GrammarOr \FUN\pmb{(}\,\MaybeIds\,\pmb{)}\; \GrammarOpt{\RETURNS \Id\;} \pmb{\{}\; \Stm\; \pmb{\}} \\
  \Exprs \GrammarDef \Expr \mid \Exprs, \Expr \\
  \ExprsMaybe \GrammarDef \varepsilon \mid \Exprs \\
  \B \GrammarDef \TRUE \mid \FALSE \\
  \GrammarOr \B\; (\AND \mid \OR)\; \B \\
  \GrammarOr \NOT\; \B \\
  \GrammarOr \Expr\; (=\: \mid\: <\: \mid\: \le\: \mid\: >\: \mid\: \ge)\; \Expr
\end{align*}
De kern van de hele grammatica draait om de volgende productieregels, die van \emph{statements}. Een statement is een programma van goede vorm. Het betekent niet nodelijkerwijs% T: per se / noodzakelijk
dat het programma \emph{valide} is, maar alle valide programma's zitten wel in $\Stm$.
\begin{align*}
  \Stm \GrammarDef \Stm; \Stm \\
  \GrammarOr \SKIP \\
  \GrammarOr \IF \B \THEN \Stm \ELSE \Stm \\
  \GrammarOr \WHILE \B \DO \Stm \\
  \GrammarOr \LOCAL \Id \\
  \GrammarOr \Id \OBJECT \\
  \GrammarOr \Id \CLONES \Id \\
  \GrammarOr \Slot = \Expr \\
  \GrammarOr \GrammarOpt{\Slot =} \Id\,\pmb{(}\,\ExprsMaybe\,\pmb{)} \\
\end{align*}
Vanwege de focus van dit werkstuk definiëren we niet precies wanneer een programma valide is en wanneer niet.
% T: Nog iets over dat we vaak \Id\ gebruiken in plaats van \Path om het ons makkelijker te maken.
% T: Is er een reden dat we "Statements" afkorten tot "Stm", "Identifiers" tot "Id", "Numbers" tot "Num" en "Expression" niet tot "Expr"? Allemaal voluit lijkt mij het handigste, dan zijn ze ook goed te gebruiken in de tekst.

% vim: syn=latex spell spl=nl cole=1 cocu=nv

\chapter{Semantisch model}

\section{Bindingen}\label{sec:bindinge}

% [Tim:] Onderstaande twee paragrafen heb ik gewijzigd zodat de termen "binding" en  "binding groep" zo gebruikt worden zoals wij dat afgesproken hebben.
Aan de basis van ons model ligt het concept van een \emph{binding}. Een binding is een toekenning van een \emph{waarde} aan een variabele (een element uit de syntactische verzameling \Id\ komt). Bindingen zijn bijvoorbeeld van belang om de gedefiniëerde variabelen binnen een scope vast te leggen, of de attributen van een bepaald object. Een \emph{groep bindingen} is een eindige functie $b : \Id \to \sV$. De verzameling van alle groepen van bindingen definiëren we dus als
% [Tim:] Hier staat dus niet meer dat gekke pijltje van \FiniteFunctions, want nu vermelden we in [Notatie en Terminologie] dat alle functies eindig zullen zijn.
$$ \sB \DEF \sV^\Id $$
% [Tim:] wat denk jij: "sectie 4.4" of "onderdeel 4.4" of gewoon "4.4" ??
We komen later terug op wat de waarden $\sV$ precies zijn in sectie \ref{sec:waarden}. Voor nu is het voldoende om te weten dat in ieder geval de natuurlijke getallen $\sN$ deel uitmaken van $\sV$.

% [Tim:] Ik geloof niet dat we in [Semantisch Model] uitvoerig in moeten gaan op het updaten van functies, aangezien dat een syntactische aangelegenheid is, die we het best kunnen beschrijving in [Notatie en Terminologie], en hier slechts even noemen.

%Een groep bindingen $b \in \sB$ is in eerste instantie leeg. Dit geven we aan met $\emptyset$. We willen natuurlijk \Id's kunnen koppelen aan waardes. Hiervoor voeren we een notatie in om $b$ te \emph{updaten}. Om bijvoorbeeld de waarde $5$ toe te kennen aan de \Id\ $x$ zoals in voorbeeld \ref{exa:todo} schrijven we
%\begin{equation*}
%  b[x \mapsto 5]
%\end{equation*}
%zodat wanneer we $x$ ``opvragen'' in $b$ we weten dat
%\begin{equation*}
%  b(x) = 5.
%\end{equation*}
%Wanneer we meerdere \Id's willen koppelen aan waardes, bijvoorbeeld $y$ aan $7$ en $z$ aan $9$ kan dat met bovenstaande notatie als volgt
%\begin{equation*}
%  \big(b[y \mapsto 7]\big)[z \mapsto 9].
%\end{equation*}
%Wat we afkorten tot
%\begin{equation*}
%  b[y \mapsto 7, z \mapsto 9].
%\end{equation*}

% [Tim:] slechts de tekst wat gereviseerd (passieve vormen, woordherhaling...)
Bindingen komen veelvuldig terug in ons model. In scopes worden \emph{variabelen} gedeclareerd en aan waarden gekoppeld. Bij objecten zijn het de \emph{attributen} die waarden krijgen toegekend.
% [Tim:] ik zou deze regel schappen
% Bij scopes moeten we ook rekening houden met eventuele bindingen in de scope buiten de huidige. Eenzelfde opzet geld voor objecten. Door prototype overerving moeten we op zoek naar een attribuut in het prototype van het huidige object, wanneer het niet gedefinieerd is in het object zelf.

\section{Scope en omliggende scopes}

% [Tim:] ik wil graag "procedureel als volgt opvatten" oid zeggen, om aan te geven dat zo'n 'procedure' niet het enige perspectief is die men kan nemen.
%Zoals in \ref{sec:voorbeelden} informeel is behandeld, zijn scopes goed te representeren met een boomstructuur. Stel we evalueren een variabele $x$ in scope $s$:
In sectie \ref{sec:voorbeelden} is informeel gebleken dat scopes conceptueel goed te zien zijn als een boomstructuur. De evaluatie van een zekere variabele \I{x} in scope $s$ zou dan procedureel als volgt kunnen worden uitgelegd:

\newCodeFragment
\codeLines{
  \codeLine{\I{x}}[We bevinden ons in een zekere scope $s$.]
}

Dan zoeken we \I{x} eerst op in de bindingen groep $b_s$, behorende bij scope $s$.
$$
  b_s(x).
$$
Zoals ook te zien in \ref{exa:todo} hebben we twee mogelijkheden:

\begin{enumerate}
  \item \I{x} is gedefinieerd in $b_s$ en we gebruiken de gevonden waarde.
  \item \I{x} is niet gedefinieerd in $b_s$ en we moeten \I{x} opzoeken in de omliggende scope.
\end{enumerate}

We moeten dus niet alleen de bindingen van de scope zelf bijhouden, maar ook een verwijzing naar zijn \emph{omgevende scope}. Een scope $s$ is definiëren we dus als een paar $(b,\pi)$, met in $b$ de bindingen en $\pi$ een \emph{verwijzing} naar de omgevende scope (ook wel \emph{parent}, of \emph{outer} scope).

We moeten benadrukken dat $\pi$ een \emph{verwijzing} is, en niet een \emph{kopie} van de bindingen groep van de omgevende scope. Stel dat we het programma in code fragment~\ref{exa:lexical} uitvoeren. Op het moment dat we $f()$ aanroepen in regel~\ref{exa:lexical:eerste} willen we dat $x$ daarna evalueert naar de waarde $2$. Evenzo moet $x$ na regel~\ref{exa:lexical:tweede} evalueren naar de waarde $4$. De scope $s_f$ van functie $f$ heeft een eigen binding $b_f$ die gedurende de executie van het programma leeg is, $x$ is namelijk niet gedeclareerd als een \LOCAL variabele. De omgevende scope $\pi_f$ van functie $f$ verwijst naar scope $s$, zodat de variabele $x$ uiteindelijk wel gevonden wordt.

\newCodeFragment[exa:lexical][Lexicale scope: opslaan en vinden van variabelen]
\codeFragmentCaption
\codeLines{
  \codeLine{\VAR \I{x}}
  \codeLine{\I{x} = 1}
  \codeLine{\VAR \I{f}}
  \codeLine[exa:lexical:def]{\I{f} = \FUN()}[Introductie nieuwe scope]
  \codeLine{\IN \I{x} = 2 \times \I{x}}
  \codeLine{}[Einde nieuwe scope]
  \codeLine[exa:lexical:eerste]{\I{f}()}[$\I{x} = 2$]
  \codeLine[exa:lexical:tweede]{\I{f}()}[$\I{x} = 4$]
}

Stel dat we geen verwijzing in de scope opslaan maar een kopie van de omgevende bindingen. Op het moment dat we $f$ definiëren in regel~\ref{exa:lexical:def} is scope $s_f$ een paar $(b_f,p_f)$ met $b_f,p_f\in\sB$. Net als hierboven zijn de eigen bindingen $b_f$ leeg. De binding $p_f$ bevat een functie onder naam $f$ en de waarde $1$ onder naam $x$. Wanneer we $x$ aanpassen door de aanroep in regel~\ref{exa:lexical:eerste} wordt dit doorgevoerd in de binding $p_f$ maar, omdat dit een kopie is, niet in de binding $b_s$ van de omgevende scope $s$. We moeten dus wel een verwijzing opslaan willen we het gevraagde gedrag krijgen. Daarnaast wordt het met kopieën erg lastig om een boomstructuur te creëren zodat we een variabele nog hogerop kunnen opzoeken.

Een scope $s$ is dus een element uit de verzameling
\begin{equation*}
  \sS \DEF \sB \times (\sL_s \cup \{\nil\}).
\end{equation*}
Hierbij zijn $\sB$ de bindingen zoals besproken in §\ref{sec:bindingen}. $\sL_s$ zijn locaties van scopes. Op het begrip locatie komen wij nog terug in §\ref{sec:locaties}. We moeten er wel rekening mee houden dat er een soort ``ultieme'' omgevende scope is. Het kan dus zijn dat een scope geen parent heeft. In dat geval zetten we
\begin{equation*}
  \pi = \nil.
\end{equation*}
We zeggen dat $\pi$ \emph{niet bestaat} of \emph{niks} is. Vandaar dat we het symbool $\nil$ toevoegen aan $\sL_s$.

\section{Objecten en prototype overerving}

In §\ref{sec:prototypen} hebben we een beeld gekregen van prototype overerving. Net als scopes en omgevende scopes, blijken objecten en prototypen te modelleren met een boomstructuur. Geheel in lijn met scopes is een object een paar met daarin zijn eigen bindingen $b$ en een verwijzing naar zijn prototype $\pi$. Natuurlijk kan een object ook geen prototype hebben. Dit geven we weer aan met $\nil$. Een object $o$ is dan een element uit
\begin{equation*}
  \sO \DEF \sB \times (\sL \cup \{\nil\}).
\end{equation*}
Hierbij zijn $\sB$ weer de bindingen uit §\ref{sec:bindingen} en $\sL_o$ zijn locaties van objecten. We maken dus een strikte scheiding tussen locaties van scopes en locaties van objecten.

\section{Waarden: referenties en primitieven}
\label{sec:waarden}

[Ze worden op dezelfde manier behandeld: objecten by-reference, dus de references zelf by-value, net als primitieven -- vandaar dat ze in dezelfde verzameling waarden zitten.]

\subsection{Natuurlijke getallen}

\subsection{Functies}

\subsection{Objecten}

\section{Locaties en geheugen}\label{sec:locaties}

\section*{Extra}

\begin{align*}
  \tag*{locaties van scopes en objecten}
  \sL &\DEF \{ (n, n) \in \sN^2 \} \\
  \tag*{functies}
  \sF &\DEF \Stm \times \Id_{\langle\rangle} \times (\Id \cup \{\nil\}) \times \sL \\
  \tag*{waarden}
  \sV &\DEF \sL \cup \sN \cup \sF \\
  \tag*{binding-verzamelingen}
  \sB &\DEF \FiniteFunctions{\sV}{\Id} \\
  \tag*{objecten}
  \sO &\DEF \sB \times (\sL \cup \{\nil\}) \\
  \tag*{scopes}
  \sS &\DEF \sB \times (\sL \cup \{\nil\}) \\
\end{align*}

% vim: spell spl=nl cole=2


\chapter{Natuurlijke Semantiek}

\section{Expressies}

\section{Statements}

\subsection{Basis}

Laten we beginnen met de simpelste constructie in onze taal, het lege statement \SKIP. Deze heeft de vorm van een axioma.

\begin{NSAxiom}{skip}
  \begin{prooftree}
    \AxiomC{$
      \Config{\SKIP}{\ms, \mo, \sigma, \tau}
      \longrightarrow
      (\ms, \mo)
    $}
  \end{prooftree}
\end{NSAxiom}

Zoals we kunnen zien zijn onze uitspraken van de vorm
\begin{equation*}
  \Config{S}{\ms,\mo,\sigma,\tau} \longrightarrow (\ms',\mo').
\end{equation*}
Hiermee bedoelen we dat
\begin{equation*}
\big(\,(S,\ms,\mo,\sigma,\tau), (\ms',\mo')\,\big) \in (\longrightarrow),
\end{equation*}
waarbij $\longrightarrow$ de volgende signatuur heeft 
\begin{equation*}
  (\longrightarrow) \subseteq (\Stm \times \MMs \times \MMo \times \LLs \times \LLo) \times (\MMs \times \MMo).
\end{equation*}
Deze transitie werkt op een statement $S\in\Stm$ in een toestand $(\ms,\mo)\in(\MMs,\MMo)$ met als extra informatie de locatie van de huidige scope $\sigma\in\LLs$ en de locatie van het huidige \THIS-object $\tau\in\MMo$. Het resultaat is een nieuwe toestand in de vorm van de twee geheugens $(\ms',\mo')\in(\MMs,\MMo)$. \SKIP\ verandert niets aan de toestand zodat $(\ms',\mo')=(\ms,\mo)$.

Voor het samenstellen van statements hebben we een regel nodig.

\begin{NSAxiom}{comp}
  \begin{prooftree}
    \AxiomC{$
      \Config{S_1}{\ms, \mo, \sigma, \tau}
      \longrightarrow
      (\ms', \mo')
    $}
    \AxiomC{$
      \Config{S_2}{\ms', \mo', \sigma, \tau}
      \longrightarrow
      (\ms'', \mo'')
    $}
    \BinaryInfC{$
      \Config{S_1; S_2}{\ms, \mo, \sigma, \tau}
      \longrightarrow
      (\ms'', \mo'')
    $}
  \end{prooftree}
\end{NSAxiom}

In dit geval geven we aan dat, wanneer we een een compositie hebben van de statements $S_1$ en $S_2$, we eerst $S_1$ uitvoeren% T: Dit is niet het juiste woord, weten even geen betere...
en daarna $S_2$. Tijdens dit proces ontstaan nieuwe toestanden, waar we natuurlijk rekening mee moeten houden. De geheugens worden dan ook netjes doorgesluisd.

Voor de controlestructuur \IF\ hebben we twee regels nodig. De eerste is voor het geval dat de \BExpr\ evalueert in \T, dan moet namelijk het statement van het \THEN-deel worden uitgevoerd. Wanneer de \BExpr\ evalueert in \F\ moet het \ELSE-deel worden uitgevoerd.
Er moet dus aan een extra voorwaarde worden voldaan om deze regels toe te mogen passen. Dit zal vaker voorkomen bij de komende deductieregels. We noteren deze extra voorwaarden onder de regel of het axioma.
%Zoals vele axioma's en deductieregels heeft dit axioma een aantal voorwaarden waaraan voldaan moet worden. Deze staan eronder genoteerd, elk op een regel.  

\begin{NSAxiom}{if$^\T$}
  \begin{prooftree}
    \AxiomC{$
      \Config{S_1}{\ms, \mo, \sigma, \tau}
      \longrightarrow
      (\ms', \mo')
    $}
    \UnaryInfC{$
      \Config{\IF b \THEN  S_1 \ELSE S_2 }{\ms, \mo, \sigma, \tau}
      \longrightarrow
      (\ms', \mo')
    $}
  \end{prooftree}
  \begin{NSConditions}
    \Cond{$ \Surr{b}^\text{B}_{\ms, \mo,\sigma,\tau} = \T $}
  \end{NSConditions}
\end{NSAxiom}

en:

\begin{NSAxiom}{if$^\F$}
  \begin{prooftree}
    \AxiomC{$
      \Config{S_2}{\ms, \mo, \sigma, \tau}
      \longrightarrow
      (\ms', \mo')
    $}
    \UnaryInfC{$
      \Config{\IF b \THEN S_1 \ELSE S_2 }{\ms, \mo, \sigma, \tau}
      \longrightarrow
      (\ms', \mo')
    $}
  \end{prooftree}
  \begin{NSConditions}
    \Cond{$ \Surr{b}^\text{B}_{\ms, \mo,\sigma,\tau} = \F $}
  \end{NSConditions}
\end{NSAxiom}

Eenzelfde tactiek passen we toe bij een \WHILE-loop.

\begin{NSAxiom}{while$^\T$}
  \begin{prooftree}
    \AxiomC{$
      \Config{S_1}{\ms, \mo, \sigma, \tau}
      \longrightarrow
      (\ms',\mo')
    $}
    \AxiomC{$
      \Config{\WHILE b \DO S_1 }{\ms', \mo', \sigma, \tau}
      \longrightarrow
      (\ms'',\mo'')
    $}
    \BinaryInfC{$
      \Config{\WHILE b \DO S_1 }{\ms, \mo, \sigma, \tau}
      \longrightarrow
      (\ms'',\mo'')
    $}
  \end{prooftree}
  \begin{NSConditions}
    \Cond{$ \Surr{b}^\text{B}_{m,\sigma,\tau} = \T $}
  \end{NSConditions}
\end{NSAxiom}

en:

\begin{NSAxiom}{while$^\F$}
  \begin{prooftree}
    \AxiomC{$
      \Config{\WHILE b \DO S_1 }{\ms, \mo, \sigma, \tau}
      \longrightarrow
      (\ms, \mo)
    $}
  \end{prooftree}
  \begin{NSConditions}
    \Cond{$ \Surr{b}^\text{B}_{\ms,\mo,\sigma,\tau} = \F $}
  \end{NSConditions}
\end{NSAxiom}

\subsection{Variabelen}

We komen nu bij een interessanter deel van de taal, namelijk het \emph{declareren} van variabelen en het \emph{toekennen} van waarden. Dit gaat allemaal over scopes

\begin{NSAxiom}{declare}
  \begin{prooftree}
    \AxiomC{$
      \Config{\VAR i}{m, \sigma, \tau}
      \longrightarrow
      m'
    $}
  \end{prooftree}
  \begin{NSConditions}
    \Cond{$ m' = m \surr{ \sigma \mapsto \big(b_{m(\sigma)}[i \mapsto \BOT], p_{m(\sigma)}\big) } $}
  \end{NSConditions}
\end{NSAxiom}

\begin{NSAxiom}{assign$^\text{i}$}
  \begin{prooftree}
    \AxiomC{$
      \Config{i = e}{m, \sigma, \tau}
      \longrightarrow
      m'
    $}
  \end{prooftree}
  \begin{NSConditions}
    \Cond{$ \sigma_\text{def} = \textsc{Find}_m(\sigma, i) $}
    \Cond{$ \Surr{e}_{m, \sigma, \tau} = v $}
    \Cond{$ m'= m \surr{ \sigma_\text{def} \mapsto \big(b_{m(\sigma_\text{def})}[ i \mapsto v ], p_{m(\sigma_\text{def})}\big) } $}
  \end{NSConditions}
\end{NSAxiom}

\subsection{Objecten}

Bij attributen gaat dit net iets anders, we hebben de hulp nodig van extra functies en voorwaarden. Zo moeten we rekening houden met het doorlopen van een pad en het speciale geval onderscheiden dat het eerste deel van het pad \THIS\ is.

\begin{NSAxiom}{assign$^\text{slot}$}
  \begin{prooftree}
    \AxiomC{$
      \Config{i.s = e}{m, \sigma, \tau}
      \longrightarrow
      m'
    $}
  \end{prooftree}
  \begin{NSConditions}
    \Cond{$ \sigma_\text{def} = \textsc{Find}_m(\sigma, i) $}
    \Cond{$ b_{m(\sigma_\text{def})}(i) = \omega \in \LL $}
    \Cond{$ \textsc{Trav}_m(\omega, s) = (\omega', j) $}
    \Cond{$ \Surr{e}_{m, \sigma, \tau} = v $}
    \Cond{$ m'= m \surr{ \omega' \mapsto \big(b_{m(\omega')}[ j \mapsto v ], p_{m(\omega')}\big) } $}
  \end{NSConditions}
\end{NSAxiom}

\begin{NSAxiom}{assign$^\text{this}$}
  \begin{prooftree}
    \AxiomC{$
      \Config{\THIS.s = e}{m, \sigma, \tau}
      \longrightarrow
      m'
    $}
  \end{prooftree}
  \begin{NSConditions}
    \Cond{$ \textsc{Trav}_m(\tau, s) = (\omega, i) $}
    \Cond{$ \Surr{e}_{m, \sigma, \tau} = v $}
    \Cond{$ m'= m \surr{ \omega \mapsto \big(b_{m(\omega)}[ i \mapsto v ], p_{m(\omega)}\big) } $}
  \end{NSConditions}
\end{NSAxiom}

\begin{NSAxiom}{object}
  \begin{prooftree}
    \AxiomC{$
      \Config{i \OBJECT}{m, \sigma, \tau}
      \longrightarrow
      m''
    $}
  \end{prooftree}
  \begin{NSConditions}
    \Cond{$ \textsc{Find}_m(\sigma, i) = \sigma_\text{def} $}
    \Cond{$ m' = m \surr{ \sigma_\text{def} \mapsto \big(b_{m(\sigma')}[i\mapsto \omega], p_{m(\sigma')}\big) } $}
    \Cond{$ m'' = m' \surr{ \omega \mapsto \big(\varnothing, \BOT\big) } $}
  \end{NSConditions}
\end{NSAxiom}

\begin{NSAxiom}{clones}
  \begin{prooftree}
    \AxiomC{$
      \Config{i \CLONES j}{m, \sigma, \tau}
      \longrightarrow
      m'
    $}
  \end{prooftree}
  \begin{NSConditions}
    \Cond{$ \Surr{i}_{m,\sigma,\tau} = \omega_i \in \LL $}
    \Cond{$ \Surr{j}_{m,\sigma,\tau} = \omega_j \in \LL $}
    \Cond{$ m' = m \surr{ \omega_i \mapsto \big(b_{m(\omega_i)}, \omega_j\big) } $}
  \end{NSConditions}
\end{NSAxiom}

\subsection{Functies}

\begin{NSAxiom}{call}
  \begin{prooftree}
    \AxiomC{$
      \Config{S_{\!f}}{m', \sigma_{\!f\text{new}}, \omega'}
      \longrightarrow
      m''
    $}
    \UnaryInfC{$
      \Config{i.s(e^*)}{m,\sigma,\tau}
      \longrightarrow
      m''
    $}
  \end{prooftree}
  \begin{NSConditions}
    \Cond{$ \sigma_\text{def} = \textsc{Find}_m(\sigma, i) $}
    \Cond{$ b_{m(\sigma_\text{def})}(i) = \omega \in \LL $}
    \Cond{$ \textsc{Trav}_n(\omega, s) = (\omega', j) $}
    \Cond{$ (S_{\!f}, I_{\!f}, i_{\!f}, \sigma_{\!f\text{def}}) = f = b_{m(\omega')}(j) $}
    \Cond{$ \sigma_{\!f\text{new}} = \textsc{Next}_\text{scope}(m) $}
    \Cond{$ m' = m\surr{ \sigma_{\!f\text{new}} \mapsto \big(\Surr{e^*}^*_{m,\sigma,\tau}(I_{\!f}), \sigma_{\!f\text{def}}\big) } $}
  \end{NSConditions}
\end{NSAxiom}

\section*{Extra}

%Deze teksten zijn vooral bedoeld als ``tekstvlees'' (lorem ipsum's). We zullen axioma's en deductieregels introduceren waarmee we de relatie $(\longrightarrow)$ definiëren, die de volgende signatuur heeft:

%$$ (\longrightarrow) \subseteq (\Stm \times \MM \times \LL \times \LL) \times \MM $$

%Wanneer we een uitspraak doen van de vorm:

%$$ \Config{S}{m,\sigma,\tau} \longrightarrow m' $$

%..dan bedoelen we daarmee dat:

%$$ \big(\,(S,m,\sigma,\tau), m'\,\big) \in (\longrightarrow) $$

Deze uitspraak moet je lezen als: ``In de toestand met geheugen $m$, scope $\sigma$ en \emph{this} object $\tau$, termineert het statement $S$, waarbij het resultaat-geheugen $m'$ is.''

Een van deze axioma's [object], heeft betrekking tot de productieregel in de grammatica die de $\OBJECT$ ``literal'' introduceert.

Wanneer bij een dergelijke opsomming van voorwaarden een nieuwe variabele wordt geïntroduceerd zoals hierboven, met de volgende vorm: $\textbf{desda } \square = \theta \dots$; dan moet deze gelezen worden als: $\textbf{desda } \exists_\theta \surr{ \square = \theta \dots }$.

% vim: syn=latex spell spl=nl cole=1 cocu=nv


\chapter{Case study: Wiskundige formulering}

De vorm van bovenstaand semantisch model geeft een conceptueel sterk beeld van hoe de taal waarschijnlijk geimplementeerd zou worden in een compiler (modulo technische details, etc\dots). Het is ook vrijwel volgens bovenstaande semantiek dat aan informatica studenten object geörienteerde talen worden uitgelegd (referenties, primitieve waarden, \dots). Je kunt echter ook semantisch model maken wat ``wiskundiger van aard'' is. Daar zal dit hoofdstuk over gaan.

Belangrijke informatie, zoals welk object van welk ander object een prototype is en hoe de scopes elkaar bevatten, maar ook welke informatie binnen een object zit opgeslagen en welke variabelen in een scope zijn gedefiniëerd, worden ditmaal door relaties bevat.

Rest nog identificatie van objecten en scopes, deze zal op eenzelfde manier worden behandeld als eerder de ``locaties'' binnen het geheugen: ze moeten identiek zijn, maar verder is het van geen belang hoe ze worden gerepresenteerd. We zullen daarom aannemen dat er twee verzamelingen identifiers zijn, $\mathbb{S}$ en $\mathbb{O}$, waarbij bovendien:

\begin{enumerate}
	\item Voor beide verzamelingen $X \in \{\mathbb{S}, \mathbb{O}\}$, bestaat er een functie $\textsc{Next}:\mathcal{P}(X) \to X$, zdd $\textsc{Next}(Y) \notin Y$ voor alle $Y \subseteq X$.
	\item $\mathbb{S} \cap \mathbb{O} = \varnothing$ (Deze eis is niet een technische benodigdheid, maar geeft wel aan dat de semantiek van deze twee soorten identifiers nu eenmaal anders is.)
\end{enumerate}

De prototype hiërarchie, eveneens de scope hiërarchie, worden natuurlijk heel goed weergegeven door partiële ordeningen. Deze twee relaties zullen zullen we als $\Proto$ en $\Outer$, respectievelijk, noteren, en aannemen:

\begin{enumerate}
	\item $\Proto$ is een (tweestemmige) partiële ordening op $\mathbb{O}$
	\item $\Outer$ is een (tweestemmige) partiële ordening op $\mathbb{S}$
\end{enumerate}

Vervolgens introduceren we de relatie $\Attr \subseteq \mathbb{O} \times \mathit{Id} \times (\mathbb{O} \cup \mathbb{P})$. De uitspraak ``$p[i] = v$'' moet worden gelezen als: $(p, i, v) \in \Attr$. Deze relatie wordt eveneens gebruikt om de ``inhoud'' van objecten weer te geven, als de graafstructuur tussen objecten. Soortgelijk definiëren we de relatie $\SDef \subseteq \mathbb{S} \times \mathit{Id} \times (\mathbb{O} \cup \mathbb{P})$, waarbij de uitspraak ``$\sigma[i] = v$'' moet worden gelezen als: $(\sigma, i, v) \in \SDef$.

Ten alle tijde moeten de prototype en scope hiërarchieën, de inhoud van objecten en scopes, en de laatste indentificaties van objecten en scopes worden bijgehouden, en dit vormt dan het ``geheugen'', of de ``toestand'': $s = (\Attr, \SDef, \Proto, \Outer, p_\mathit{next}, \sigma_\mathit{next})$.

\begin{align*}
	\left<\SKIP, s, \tau, \sigma\right> &\longrightarrow s \\
	\left<x = e, s, \tau, \sigma\right> &\longrightarrow s[\; \sigma_d[x] = e \;] \\
	& \textbf{desda}\; \sigma_d = \max \{\, \sigma' \in \mathbb{S} \mid \sigma'[x] \land \sigma \OuterEq \sigma' \,\} \\
	\left<x \OBJECT, \tau, \sigma\right> &\longrightarrow s[\; \sigma_d[x] = p \;][\; p_\mathit{next} \mapsto p + 1 \;] \\
	& \textbf{desda}\; \sigma_d = \max \{\, \sigma' \in \mathbb{S} \mid \sigma'[x] \land \sigma \OuterEq \sigma' \,\} \\
	& \textbf{en}\; p = {p_\mathit{next}}_s
\end{align*}


\backmatter

\end{document}

% vim: spell spl=nl
