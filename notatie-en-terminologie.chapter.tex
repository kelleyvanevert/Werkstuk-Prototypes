
\chapter{Notatie en terminologie}

\section{Beschouwing semantisch model}

We definieren in dit werkstuk een natuurlijke semantiek, d.w.z.~een ?-ste orde logica, met axioma's en deductieregels, en een bijbehorende structuur waarin deze zich afspeelt.

Deze structuur, die we ook wel het \emph{semantisch model} zullen noemen, heeft onderstaand opgesomde elementen. Deze worden verderop precies gedefinieerd, onderstaande opsomming geeft slechts een algemeen beeld.

\begin{description}
	\item[$\mathbb{M}$]\hfill\\ De verzameling mogelijke \emph{geheugens}, welke ook wel als \emph{eindtoestanden} worden geinterpreteerd.
	\item[$(\mathit{Stm} \times \mathbb{M} \times \mathbb{L} \times \mathbb{L})$]\hfill\\ De verzameling \emph{toestanden}, ook wel \emph{configuraties}.
	\item[$(\longrightarrow)$]\hfill\\ Een tweeplaatsig predikaat welke als eerste argument een element uit de verzameling van toestanden neemt, en als tweede argument een element uit de verzameling van eindtoestanden $(\mathbb{M}\dots)$. De uitspraak $(S, m, \sigma, \tau) \longrightarrow m'$ moet worden geinterpreteerd worden als:
	\begin{quote} ``Het programma $S$, met geheugen $m$, in scope $\sigma$ en met als $\mathbf{this}$ object $\tau$, resulteert in eindtoestand $m'$, mits $S$ \emph{correct} is''. \end{quote}
\end{description}

\section{Notationele conventies}

Terwille van elegantie houden we een aantal gebruikelijke notationele conventies aan:

\begin{enumerate}
	\item Voor elke twee willekeurige tweestemmige predikaten $\mathsf{S}$ en $\mathsf{T}$ (mogelijk ook $=$), en drie willekeurige elementen $a$, $b$ en $c$, definieren we de afkorting: $$a \operatorname{\mathsf{S}} b \operatorname{\mathsf{T}} c \buildrel{\mathrm{def}}\over{=} a \operatorname{\mathsf{S}} b \land b \operatorname{\mathsf{T}} c$$ in het geval dat deze bewering correct getypeerd is.
	\item Op eenzelfde manier definieren we ook de volgende afkorting: $$ \{a \in A \mid \phi \} \buildrel{\mathrm{def}}\over{=} \{a \mid a \in A \mid \phi\}$$
\end{enumerate}

[...]
