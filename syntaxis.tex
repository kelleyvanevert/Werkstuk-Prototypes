\chapter{Syntaxis}

\section{Grammatica}

\begin{tabular}{lcl}
\syn{Statement}  &::=& \syn{Statement} $;$ \syn{Statement} \\
                  &|&  \SKIP \\
                  &|&  \IF \syn{Test} \THEN \syn{Statement} \ELSE \syn{Statement} \\
                  &|&  \WHILE \syn{Test} \DO \syn{Statement} \\
                  &|&  \PRINT \syn{Expression} \\
                  &|&  \VAR \syn{Slot} \{ $,$ \syn{Slot} \} \\
                  &|&  \syn{Slot} $=$ \syn{Expression} \\
                  &|&  \FUN \syn{Slot} $($ [ \syn{Identifier} \{ , \syn{Identifier} \} ] $)$
                       [\RETURNS \syn{Slot} ] \IS \syn{Statement} \\
                  &|&  [ \syn{Slot} $=$ ] \syn{Slot}
                       $($ [ \syn{Expression} \{ $,$ \syn{Expression} \} ] $)$ \\
                  &|&  \OBJ \syn{Slot} [\CLONES \syn{Slot} ] \\
                  \\
\syn{Slot}       &::=& \syn{Identifier} \{ $.$ \syn{Identifier} \} \\
                  \\
\syn{Identifier} &::=& ( a | \dots | z | A | \dots | Z | \_ )+ \\
                  \\
\syn{Expression} &::=& \syn{Number} \\
                  &|&  \syn{Slot} \\
                  &|&  \syn{Expression} \syn{Operator} \syn{Expression} \\
                  \\
\syn{Operator}   &::=& $+$ | $-$ | $\times$ | $/$ \\
                  \\
\syn{Number}     &::=& ( 0 | \dots | 9 )+ \\
                  \\
\syn{Test}       &::=& \syn{Boolean} \\
                  &|&  \syn{Test} $\wedge$ \syn{Test} \\
                  &|&  \syn{Test} $\vee$ \\
                  &|&  $\lnot$ \syn{Test} \\
                  &|&  \syn{Expression} \syn{Relation} \syn{Expression} \\
                  \\
\syn{Relation}   &::=& $=$ | $\neq$ | $<$ | $\le$ | $>$ | $\ge$ \\
                  \\
\syn{Boolean}    &::=& \TRUE\ | \FALSE
\end{tabular}

\section{Een programma over deuren}

\begin{program}
obj Door clones Object       # Maak een nieuwe deur

fun Door.open() is           # Een functie om de deur te openen
    print 42

obj LockedDoor clones Door   # Maak een nieuwe gesloten deur...
var LockedDoor.isLocked
LockedDoor.isLocked = 1      # ...die dicht is

fun LockedDoor.unlock(code) is
    if code == this.code then# Als de code correct is...
        isLocked = 0         # ...openen we de deur

fun LockedDoor.open() is     # Herdefinieer om isLocked te gebruiken
    if isLocked == 1 then
        print 0
    else
        proto.open()         # Expliciete aanroep naar prototype

obj Safe clones LockedDoor   # Maak een kluis...
var Safe.code
Safe.code = 1234             # ...en ken een code toe

Safe.unlock(4321)
Safe.open()                  # Geeft 0, de deur is gesloten
Safe.unlock(1234)
Safe.open()                  # Geeft 42, we de juiste code hadden
\end{program}

\section{Een programma over personen}

\begin{program}
obj Person clones Object

var Person.total
Person.total = 0

fun Person.initialize() is   # Wordt aangeroepen bij iedere kloon
    Person.total = Person.total + 1

obj Me clones Person
var Me.age
Me.age = 37

obj You clones Person
var You.age
You.age = 42

print Person.total           # Geeft 2
\end{program}

\section{Een programma over scope}

\begin{program}
var x
x = 1
fun g() is
    print x
    x = 2
fun f() is
    var x
    x = 3
    g()
f()
print x
\end{program}

\section{Een programma over counters}

\begin{minipage}{0.5\textwidth}
\begin{program}
fun counter(n) returns next is

    fun next() returns n is
        n = n + 1

var c
c = counter(5)

var d
d = counter(42)

var i, j
i = c() # i = 6
j = d() # j = 43
\end{program}
\end{minipage}
\begin{minipage}{0.5\textwidth}
\begin{program}
obj Counter clones Object
var Counter.n
fun Counter.next() is
    n = n + 1

obj c clones Counter
c.n = 5

obj d clones Counter
d.n = 42


c.next() # c.n = 6
d.next() # d.n = 43
\end{program}
\end{minipage}

% vim: ft=context
