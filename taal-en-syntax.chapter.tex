
\chapter{Taal en syntax}

In dit hoofdstuk zullen we de taal presenteren waarvoor we een natuurlijk taal construeren. De taal maakt gebruikt van prototype overerving en lexicaal bereik. Eerst zullen we een aantal voorbeeldprogramma's beschouwen, om zo informeel het karakter van de te formaliseren taal over te brengen. Daarna geven we een rigoreuze definitie met behulp van een BNF grammatica. De structuur van de productieregels van grammatica worden in latere hoofdstukken gebruikt om axioma's en deductieregels op te stellen. Daarmee heeft de grammatica in zekere zin een dubbele functie.

Elk voorbeeldprogramma en zijn toelichtingen worden als volg gepresenteerd:

	\begin{SyntaxExample}
		\N \VAR f & \textit{- variabelen moeten worden gedeclareerd} \\
		\N f = \FUN(i) \RETURNS n \\
		\N \IN \VAR n \\
		\N \IN n = 2 \times (i + 5) \\
		\N & \textit{- x bestaat niet in deze scope} \\
		\N \VAR x & \textit{- x is ongedefinieerd (maar wel aanwezig)} \\
		\N x = f(42) & \textit{- x = 89}
	\end{SyntaxExample}

De toelichtingen moeten als informeel commentaar worden beschouwd, waarmee we aan proberen te geven hoe het programma zich gedraagt. Vaak zijn het uitspraken over de toestand waarin het programma zich bevindt, direct na de linker regel te hebben ``uitgevoerd''.

\section{Voorbeeldprogramma's}

Een variabele moet gedeclareerd worden, en pas daarna kan er een waarde aan worden toegekend.

	\begin{SyntaxExample}
		\N & \textit{- x bestaat niet (in deze scope)} \\
		\N \VAR x & \textit{- x is ongedefinieerd (maar wel aanwezig)} \\
		\N x = 5 & \textit{- x = 5}
	\end{SyntaxExample}

Het concept van declaratie is juist in deze taal, gezien het lexicaal bereik van variabelen, heel belangrijk. Vergelijk het bovenstaande programma fragment bijvoorbeeld met de volgende situatie.

Variabelen hebben geen vaste type. Er zijn drie typen waarden in de taal: getallen, functies en objecten.

	\begin{SyntaxExample}
		\N \VAR x \\
		\N x = 5 & \textit{- de waarde van } x \textit{ is een getal} \\
		\N x = \FUN()\,\{\;\SKIP\;\} & \textit{- de waarde van x is een functie} \\
		\N x \OBJECT & \textit{- de waarde van x is een object}
	\end{SyntaxExample}

De taal is object georienteerd.

	\begin{SyntaxExample}
		\N \VAR o \\
		\N o \OBJECT \\
		\N & \textit{- o.f is niet gedefinieerd} \\
		\N o.f = \FUN()\,\{\;\SKIP\;\} & \textit{- toekenning waarde aan object attribuut} \\
		\N & \textit{- o.f is wel gedefinieerd} \\
		\N o.n = 5 &
	\end{SyntaxExample}

Van de drie typen, zijn getallen en functies \emph{primitief}, en objecten \emph{niet primitief}. Primitieve waarde worden zelf gekopieerd (\emph{by-value}), maar van niet-primitieve waarden worden \emph{referenties} gekopieerd (\emph{by-reference}).

	\begin{SyntaxExample}
		\N \VAR x;\; x = 6 \\
		\N \VAR y;\; y = x & \textit{- x = 6 en y = 6} \\
		\N y = 7 & \textit{- x = 6 en y = 7} \\
		\N \\
		\N \VAR p;\; p.n = 6 \\
		\N \VAR q;\; q = p & \textit{- p en q verwijzen nu naar hetzelfde object} \\
		\N & \textit{- p.n = 6 en q.n = 6} \\
		\N q.n = 7 & \textit{- p.n = 7 en q.n = 7}
	\end{SyntaxExample}

\subsection{Lexical scope}

Als in een zekere scope een variabele wordt gereferenceerd (nog) niet is gedefinieerd, wordt in omliggende scopes ``gezocht'' naar een definitie van deze variabele.

\begin{SyntaxExample}\label{exa:lexical}
		\N \VAR x; \\
		\N \VAR f;\; f = \FUN(i) \\
		\N \IN x = i + 5 \\
		\N \\
		\N f(5) & \textit{- x = 10}
	\end{SyntaxExample}

..maar wanneer deze wel in de huidige scope bestaat, worden omliggende scopes ``met rust gelaten''.

	\begin{SyntaxExample}
		\N \VAR x \\
		\N \VAR f \\
		\N f = \FUN(i) \\
		\N \IN \VAR x \\
		\N \IN x = i + 5 \\
		\N \\
		\N f(5) & \textit{- x heeft nog geen waarde}
	\end{SyntaxExample}

Telkens wanneer een functie wordt aangeroepen, wordt een \emph{nieuwe scope} aangemaakt voor lokale variabelen. Variabelen van deze nieuwe scope kunnen later nog gereferenceerd worden, doordat bijvoorbeeld de functie een lokale functie teruggeeft.

	\begin{SyntaxExample}
		\N \VAR f \\
		\N f = \FUN(n) \RETURNS g \\
		\N \IN \VAR g \\
		\N \IN g = \FUN() \RETURNS n \\
		\N \IN \IN n = n + 1 \\
		\N \\
		\N \VAR c \\
		\N c = f(5) & \textit{- c() $\rightarrow$ 6, 7, 8, \dots}
	\end{SyntaxExample}

\begin{lstlisting}[caption=Een countervoorbeeld,label=exa:counter]
local f
f = function(n) returns g
    local g
    g = function() returns n
        n = n + 1

local c
c = f(5)
c()                          # $\rightarrow 6, 7, 8, \dots$
\end{lstlisting}

\fbox{maar dan wat beter geschreven, etc...}

\subsection{Prototype overerving}

Prototype overerving is een eenvoudige en dynamische variant van object-geörienteerd programmeren. Net als in klassieke object-gebaseerde talen is er sprake van een object waarin \emph{attributen} zijn gedefinieerd. Elk object heeft ook een expliciete \emph{ouder} (of \enquote{parent}). Wanneer we binnen een object een attribuut willen evalueren, doen we dit in drie stappen:
\begin{enumerate}
  \item Bekijk of het attribuut gedefinieerd is in het object zelf. In dat geval weten we de waarde en leveren deze op.
  \item Anders zoeken we het attribuut op in de ouder van het object. Ook dan weten we de waarde en leveren deze op.
  \item Wanneer ook de ouder het attribuut niet bevat, herhalen we de zoektocht voor alle volgende ouders totdat we het attribuut hebben gevonden.
\end{enumerate}
Ook hier is dus sprake van een boomstructuur.

Het grote verschil tussen object-gebaseerde talen en prototype-gebaseerde talen is dat de tweede geen onderscheid maakt tussen \emph{klassen} en \emph{instanties}. Een prototype heeft beide functies. Neem bijvoorbeeld het prototype @Deur@:
\begin{lstlisting}[name=deuren]
local Deur
Deur object
\end{lstlisting}
We kunnen @Deur@ direct als instantie gebruiken door een attribuut te zetten:
\begin{lstlisting}[name=deuren]
Deur.open = 1
\end{lstlisting}
Een @Deur@ is standaard open. Maar we kunnen @Deur@ ook als een klasse gebruiken door ervan te erven. In prototype-gebaseerde talen heet dit \emph{klonen}:
\begin{lstlisting}[name=deuren]
local GeslotenDeur
GeslotenDeur object
GeslotenDeur clones Deur
\end{lstlisting}
@GeslotenDeur@ heeft dan alle attributen van @Deur@:
\begin{lstlisting}[name=deuren]
GeslotenDeur.open            # => 1
\end{lstlisting}
Maar een @GeslotenDeur@ moet natuurlijk gesloten zijn. We zetten zijn attribuut @open@ op @0@:
\begin{lstlisting}[name=deuren]
GeslotenDeur.open = 0
\end{lstlisting}
Een gewone @Deur@ is nog steeds open:
\begin{lstlisting}[name=deuren]
Deur.open                    # => 1
\end{lstlisting}
Attributen worden dus per object bewaard. Door @open@ op @0@ te zetten in @GeslotenDeur@ verandert er niks in @Deur@.

We kunnen net zoveel klonen maken van een object als we willen en net zo diep klonen als we willen. Neem een @GlazenDeur@, dit is natuurlijk ook een @Deur@, maar wel doorzichtig:
\begin{lstlisting}[name=deuren]
local GlazenDeur
GlazenDeur object
GlazenDeur clones Deur
GlazenDeur.doorzichtig = 1
\end{lstlisting}
Een gewone @Deur@ heeft het attribuut @doorzichtig@ niet, en dus een @GeslotenDeur@ ook niet:
\begin{lstlisting}[name=deuren]
GeslotenDeur.doorzichtig     # => fout!
\end{lstlisting}
Maar we kunnen besluiten dat deuren standaard niet doorzichtig zijn:
\begin{lstlisting}[name=deuren]
Deur.doorzichtig = 0
\end{lstlisting}
Zodat ook onze @GeslotenDeur@ niet doorzichtig is:
\begin{lstlisting}[name=deuren]
GeslotenDeur.doorzichtig     # => 0
\end{lstlisting}
Maar er geld nog steeds:
\begin{lstlisting}[name=deuren]
GlazenDeur.doorzichtig       # => 1
\end{lstlisting}

We zien dat we met prototypes een zeer flexibele methode hebben om object-geörienteerd te programmeren. Het is niet nodig om de compiler of parser van te voren uit te leggen dat objecten aan bepaalde \enquote{blauwdrukken} moeten voldoen. We creëren objecten \enquote{on-the-fly}, alsmede hun attributen en relaties. Deze methode komt terug in talen als JavaScript, IO en Self.

Natuurlijk is het ook mogelijk om \emph{methoden} te definiëren. Dit zijn functies die gekoppeld zijn aan een specifiek object. Stel dat we een @GeslotenDeur@ graag open willen maken. We definiëren:
\begin{lstlisting}[name=deuren]
GeslotenDeur.ontsluit = function (poging)
    if poging == this.code then
        this.open = 1
    else
        this.open = 0
\end{lstlisting}
@this@ is hier een expliciete verwijzing naar het huidige object. Op dit moment kunnen we @ontsluit@ nog niet aanroepen op @GeslotenDeur@:
\begin{lstlisting}[name=deuren]
GeslotenDeur.ontsluit(1234)  # => fout!
\end{lstlisting}
Het attribuut @code@ is immers niet gedefinieerd in @GeslotenDeur@ noch in zijn prototype @Deur@.

We kunnen natuurlijk een @code@ toekennen aan @GeslotenDeur@, maar laten we een specifieke @GeslotenDeur@ maken met een @code@:
\begin{lstlisting}[name=deuren]
local Kluis
Kluis object
Kluis clones GeslotenDeur
Kluis.code = 4321
\end{lstlisting}
Wanneer we de methode @ontsluit@ aanroepen is deze niet gedefinieerd in @Kluis@, maar wel in zijn prototype @GeslotenDeur@. Die wordt dan uitgevoerd. Een belangrijke observatie is dat @ontsluit@ wel wordt aangeroepen op @Kluis@. Dat betekent dat @this@ verwijst naar @Kluis@ en niet @GeslotenDeur@. Het attribuut @code@ wordt dan wel gevonden!
\begin{lstlisting}[name=deuren]
Kluis.ontsluit(1234)
Kluis.open                   # => 0
\end{lstlisting}
Oeps\dots Dat was de verkeerde code, nog een poging:
\begin{lstlisting}[name=deuren]
Kluis.ontsluit(4321)
Kluis.open                   # => 1
\end{lstlisting}

\section{Grammatica}

[...en vervolgens helemaal formeel -- even uitleggen van BNF etc..]

% vim: spell spl=nl
