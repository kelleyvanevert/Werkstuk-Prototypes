
\chapter{Taal en syntaxis}

In dit hoofdstuk presenteren we de taal presenteren waarvoor we een natuurlijk semantiek construeren. De taal maakt gebruikt van prototype overerving en lexicaal bereik. Eerst beschouwen we een aantal voorbeeldprogramma's, om zo informeel de te formaliseren taal te karakteriseren. Daarna geven we een rigoreuze definitie met behulp van een \BNF\ grammatica.

De structuur van de productieregels van deze grammatica worden in latere hoofdstukken gebruikt om axioma's en deductieregels op te stellen. Daarmee heeft de grammatica in zekere zin een dubbele functie.

Het is belangrijk om te vermelden dat het ons hierbij niet gaat om een ``goed uitziende'' taal te maken, maar enkel om er de essentiële onderdelen in te verwerken die nodig zijn voor ons doeleinde: het formaliseren van lexicaal bereik en prototype overerving met een natuurlijke semantiek. Om diezelfde reden moet de syntaxis van de taal voornamelijk worden beschouwd als wat een mogelijke representatie van een \emph{abstract syntax tree} van een ``echte'' programmeertaal zou kunnen zijn.

\section{Voorbeeldprogramma's}
\label{sec:voorbeelden}

Elk voorbeeldprogramma en zijn toelichtingen worden als volg gepresenteerd:

\newCodeFragment[exa:template][Het eerste voorbeeldprogramma]
\codeFragmentCaption
\codeLines{
  \codeLine{\VAR \id{f}}[$\id{f}$ moet eerst worden gedefiniëerd]
  \codeLine{\id{f} = \FUN(i) \RETURNS \id{n}}
  \codeLine{\IN \VAR \id{n}}
  \codeLine{\IN \id{n} = 2 \times (\id{i} + 5)}
  \codeLine{}[$\id{x}$ bestaat niet in deze scope]
  \codeLine{\VAR \id{x}}[\id{x} heeft nog geen waarde, maar is wel gedefiniëerd]
  \codeLine{\id{x} = \id{f}(42)}[\id{x} heeft nu de waarde 94]
}

De toelichtingen moeten als informeel commentaar worden beschouwd, waarmee we aan proberen te geven hoe het programma zich gedraagt. Vaak zijn het uitspraken over de toestand waarin het programma zich bevindt, direct na de linker regel te hebben ``uitgevoerd''.

\subsection{Basis}

\subsubsection{Declaratie van variabelen}

Een variabele moet altijd eerst worden gedeclareerd, alvorens er een waarde aan kan worden toegekend of het op andere manieren kan worden gebruikt. Een programma waarin variabelen worden gereferenceerd die nooit zijn gedefiniëerd is niet valide. In code fragment \ref{exa:declaration} zie je hoe declaratie plaatsvindt.

\newCodeFragment[exa:declaration][Declaratie van variabelen]
\codeFragmentCaption
\codeLines{
  \codeLine{}[\id{x} bestaat (nog) niet]
  \codeLine{\VAR \id{x}}[\id{x} heeft nog geen waarde, maar is wel gedefiniëerd]
  \codeLine{\id{x} = 5}[\id{x} bevat nu de waarde 5]
}

Het concept van declaratie is juist in deze taal heel belangrijk, gezien het lexicaal bereik van variabelen. Wat lexicaal bereik precies inhoudt wordt weldra behandeld.

\subsubsection{Types}

Variabelen kunnen na declaratie waarden aannemen. Onze taal bevat waarden van drie types (natuurlijk getal, functie, object), en het onderscheid tussen deze types wordt op \emph{dynamisch} niveau gemaakt i.p.v~op syntactisch niveau. Dat houdt in dat een willekeurige variabele elke willekeurige waarde kan aannemen, van elk willekeurig type. Ook kan het in zijn levensduur waarden van meerder types bevatten. Code fragment \ref{exa:types} geeft dit idee weer.

\newCodeFragment[exa:types][Waarden van verschillende types]
\codeFragmentCaption
\codeLines{
  \codeLine{\VAR \id{x}}[declaratie zonder type indicatie]
  \codeLine{\id{x} = 5}[type ``natuurlijk getal'']
  \codeLine{\id{x} = \FUN(\;)\{\dots\}}[type ``functie'']
  \codeLine{\id{x} \OBJECT}[type ``object'', deze rare initiälisatie heeft een reden]
}

\subsection{Lexicaal bereik}

De \emph{scope} (ook wel \emph{bereik}) van een variabele in een programma, is het deel van het programma waarin zij kan worden gebruikt. En zijn verschillende manieren om dit bereik te definiëren, een daarvan is \emph{lexicaal bereik} (ook wel \emph{lexical} of \emph{static scoping}) er één.

% [Tim:] ik kan dit niet echt mooi verwoorden ...?
Gegeven een bepaalde definitie van dit bereik, is de vraag: Als ik een variabele naam tegenkom, over welke variabele heb ik het dan? Code fragmenten \ref{exa:zoek} en \ref{exa:zoek2} illustreren deze vraag. Met \emph{lexicaal bereik} wordt deze dus ruwweg in de ``lexicale omgeving'' van de referentie gezocht.

\newCodeFragment[exa:zoek][Zoek de definitie van variabele \id{x}]
\codeFragmentCaption
\codeLines{
  \codeLine{\VAR \id{x}}[dit is de gezochte definitie van \id{x}]
  \codeLine{\id{x} = 42}
  \codeLine{}
  \codeLine{\VAR \id{f}}
  \codeLine{\id{f} = \FUN(\;)}
  \codeLine{\IN \dots \id{x} \dots}[waar is deze \id{x} gedefiniëerd?]
}

\newCodeFragment[exa:zoek2][Zoek de definitie van variabele \id{x}]
\codeFragmentCaption
\codeLines{
  \codeLine{\VAR \id{x}}
  \codeLine{\id{x} = 42}
  \codeLine{}
  \codeLine{\VAR \id{f}}
  \codeLine{\id{f} = \FUN(\;)}
  \codeLine{\IN \VAR \id{x}}[dit is de gezochte definitie van \id{x}]
  \codeLine{\IN \id{x} = 43}
  \codeLine{\IN \dots \id{x} \dots}[waar is deze \id{x} gedefiniëerd?]
}

Als je de code zó indenteert zoals in bovenstaande twee code fragmenten (\ref{exa:zoek} en \ref{exa:zoek2}): bij elke functie definitie wordt een regel geïndenteerd, dan komt de indentatie ongeveer overeen met de boomstructuur van de scopes. Het moge duidelijk zijn dat als men ``naar buiten toe'' zoekt naar de definities van variabelen, een zekere boomstructuur van toepassing is.

Nieuwe scopes ontstaan in onze taal enkel (en altijd) bij functie applicatie. Een functie op zich is een \emph{primitieve waarde}, wat betekent dat er ``niks gebeurt'' als een functie wordt gedefiniëerd, net als er niks gebeurt wanneer je een getal aan een variabele toekent. Bij functie applicatie, echter, wordt een nieuwe scope aangemaakt, met als \emph{omliggende scope} de scope waarin de functie origineel is gedefiniëerd. Daarin wordt vervolgens de \emph{body} van de functie uitgevoerd. In code fragment \ref{exa:counting} wordt het belang van dit proces weergegeven: als nieuwe scopes worden aangemaakt bij de definitie van functies, zou de uitvoer van $\id{d}(\;)$ 8 zijn i.p.v.~43.

\newCodeFragment[exa:counting][Het belang van creatie van scopes \emph{bij functie applicatie}]
\codeFragmentCaption
\codeLines{
  \codeLine{\VAR \id{f}}
  \codeLine{\id{f} = \FUN(n) \RETURNS \id{g}}
  \codeLine{\IN \VAR \id{g}}
  \codeLine{\IN \id{g} = \FUN(\;) \RETURNS \id{n}}
  \codeLine{\IN \IN \id{n} = \id{n} + 1}
  \codeLine{}
  \codeLine{\VAR \id{c}}
  \codeLine{\id{c} = \id{f}(5)}
  \codeLine{\id{c}()}[de eerste aanroep $\id{c}(\;)$ levert eerst 6 op\dots]
  \codeLine{\id{c}()}[de tweede aanroep $\id{c}(\;)$ levert eerst 7 op\dots]
  \codeLine{}[]
  \codeLine{\id{d} = \id{f}(42)}[$\id{d}(\;)$: 43, 44, 45, 46, \dots]
}

\subsection{Prototype overerving}

Prototype overerving is een variant van object-geörienteerd programmeren. De kern van object-geörienteerd programmeren is het concept van een \emph{object}, dat ertoe dient een verschijnsel uit de werkelijkheid na te bootsen (een reëel object, een patroon, een abstract idee). Het doel is om meer te kunnen programmeren op een conceptueel niveau. Daarmee wordt bijvoorbeeld zowel creatie als onderhoud van de code makkelijker.

Veel objecten zullen natuurlijk gelijke eigenschappen vertonen, of dezelfde structuur hebben. Verder wilt men concepten als specificering en generalisering toepassen op objecten. Deze problemen kunnen op meerdere manieren worden aangepakt. De bekendste variant is \emph{klasse gebaseerde} object-oriëntatie (ook wel \emph{klassieke object-oriëntatie}) en richt zich op het concept van een \emph{klasse}. Objecten van een bepaalde klasse vertonen de structuur en gedrag van die klasse en heten \emph{instanties}. Van specificering is sprake als een klasse eigenschappen van een andere klasse \emph{overerft}. Klassieke object-oriëntatie vind men in talen als Java en C\#.

Een andere aanpak met hetzelfde doel is \emph{prototype gebaseerde} object-oriëntatie. Daarbij wordt geen scheiding gemaakt tussen de concepten klasse, die structuur en gedrag specificeert, en instantie, die enkel deze eigenschappen vertoont. In plaats daarvan wordt gewerkt met een prototype structuur, waarbij elk object naar een bepaald \emph{prototype}-object refereert. Nu zijn objecten zelf de dragers van structuur en gedrag.
%Deze methode kan als flexibeler worden gezien, maar ook als een wat minder reëel beeld van de werkelijkheid worden opgevat.

Technisch gezien werkt prototype overerving als volgt. Van elk object is een prototype bekend, of het heeft geen prototype. Wanneer men een attribuut opvraagt van een zeker object, kan de op te leveren waarde procedureel als volgt worden opgevat:

\begin{enumerate}
  \item Bekijk of het attribuut gedefiniëerd is in het object zelf. In dat geval weten we de waarde en leveren deze op.
  \item Anders zoeken we het attribuut op in het prototype van het object. Ook dan weten we de waarde en leveren deze op.
  \item Wanneer ook het prototype het attribuut niet bevat, herhalen we de zoektocht voor alle volgende prototypen totdat we het attribuut hebben gevonden.
\end{enumerate}

Het grote verschil tussen object-gebaseerde talen en prototype-gebaseerde talen is dus dat de tweede geen onderscheid maakt tussen klassen en instanties. Een prototype heeft beide functies. Neem bijvoorbeeld het object \id{Deur}:

\newCodeFragment

\codeLines{
  \codeLine{\VAR \id{Deur}}
  \codeLine{\id{Deur} \OBJECT}
}

We kunnen \id{Deur} direct als instantie gebruiken door een attribuut te zetten:

\codeLines{
  \codeLine{\id{Deur}.\id{open} = 1}
}

Een \id{Deur} is standaard open. We kunnen \id{Deur} ook als een prototype gebruiken. In prototype-gebaseerde talen heet dit \emph{klonen}:

\codeLines{
  \codeLine{\VAR \id{GeslotenDeur}}
  \codeLine{\id{GeslotenDeur} \OBJECT}
	\codeLine{\id{GeslotenDeur} \CLONES \id{Deur}}
}

\id{GeslotenDeur} heeft dan alle attributen van \id{Deur}:

\codeLines{
  \codeLine{\id{GeslotenDeur}.\id{open}}[waarde $\to$ 1]
}

Maar een \id{GeslotenDeur} moet natuurlijk gesloten zijn. We zetten zijn attribuut \id{open} op @0@:

\codeLines{
	\codeLine{\id{GeslotenDeur}.\id{open} = 0}
}

Een gewone \id{Deur} is nog steeds open:

\codeLines{
  \codeLine{\id{Deur}.\id{open}}[waarde $\to$ 1]
}

Attributen worden dus per object bewaard. Door \id{open} op @0@ te zetten in \id{GeslotenDeur} verandert er niks in \id{Deur}.

We kunnen net zoveel klonen maken van een object als we willen en net zo diep klonen als we willen. Neem een \id{GlazenDeur}, dit is natuurlijk ook een \id{Deur}, maar wel doorzichtig:

\codeLines{
  \codeLine{\VAR \id{GlazenDeur}}
  \codeLine{\id{GlazenDeur} \OBJECT}
  \codeLine{\id{GlazenDeur} \CLONES \id{Deur}}
  \codeLine{\id{GlazenDeur}.\id{doorzichtig} = 1}
}

Een gewone \id{Deur} heeft het attribuut \id{doorzichtig} niet, en dus een \id{GeslotenDeur} ook niet:

\codeLines{
  \codeLine{\id{GeslotenDeur}.\id{doorzichtig}}[fout!]
}

Maar we kunnen besluiten dat deuren standaard niet doorzichtig zijn:

\codeLines{
  \codeLine{\id{Deur}.\id{doorzichtig} = 0}
}

Zodat ook onze \id{GeslotenDeur} niet doorzichtig is:

\codeLines{
  \codeLine{\id{GeslotenDeur}.\id{doorzichtig}}[waarde $\to$ 0]
}

Maar er geld nog steeds:

\codeLines{
  \codeLine{\id{GlazenDeur}.\id{doorzichtig}}[waarde $\to$ 1]
}

We zien dat we met prototypes een zeer flexibele methode hebben om object-geörienteerd te programmeren. Het is niet nodig om de compiler of parser van te voren uit te leggen dat objecten aan bepaalde ``blauwdrukken'' moeten voldoen. We creëren objecten ``on-the-fly'', alsmede hun attributen en relaties. Deze methode komt terug in talen als JavaScript, IO en Self.

Natuurlijk is het ook mogelijk om \emph{methoden} te definiëren. Dit zijn functie attributen gekoppeld aan een specifiek object. Stel dat we een \id{GeslotenDeur} graag open willen maken. We definiëren:

\codeLines{
  \codeLine{\id{GeslotenDeur}.\id{ontsluit} = \FUN (\id{poging})}
  \codeLine{\IN \IF (\id{poging} = \THIS.\id{code}) \THEN}
  \codeLine{\IN \IN \THIS.\id{open} = 1}
  \codeLine{\IN \ELSE}
  \codeLine{\IN \IN \THIS.\id{open} = 0}
}

\id{this} is hier een expliciete verwijzing naar het huidige object. Op dit moment kunnen we \id{ontsluit} nog niet aanroepen op \id{GeslotenDeur}:

\codeLines{
  \codeLine{\id{GeslotenDeur}.\id{ontsluit}(1234)}[fout!]
}

Het attribuut \id{code} is immers niet gedefinieerd in \id{GeslotenDeur} noch in zijn prototype \id{Deur}.

We kunnen natuurlijk een \id{code} toekennen aan \id{GeslotenDeur}, maar laten we een specifieke \id{GeslotenDeur} maken met een \id{code}:

\codeLines{
  \codeLine{\VAR \id{Kluis}}
  \codeLine{\id{Kluis} \OBJECT}
  \codeLine{\id{Kluis} \CLONES \id{GeslotenDeur}}
  \codeLine{\id{Kluis}.\id{code} = 4321}
}

Wanneer we de methode \id{ontsluit} aanroepen is deze niet gedefinieerd in \id{Kluis}, maar wel in zijn prototype \id{GeslotenDeur}. Die wordt dan uitgevoerd. Een belangrijke observatie is dat \id{ontsluit} wel wordt aangeroepen op \id{Kluis}. Dat betekent dat \id{this} verwijst naar \id{Kluis} en niet \id{GeslotenDeur}. Het attribuut \id{code} wordt dan wel gevonden:

\codeLines{
  \codeLine{\id{Kluis}.\id{ontsluit}(1234)}
  \codeLine{\id{Kluis}.\id{open}}[waarde $\to$ 0]
}

Helaas was dat de verkeerde code, we proberen het nog een keer:

\codeLines{
  \codeLine{\id{Kluis}.\id{ontsluit}(1234)}
  \codeLine{\id{Kluis}.\id{open}}[waarde $\to$ 1]
}

%TODO: Meerdere overerving niet mogelijk?

\section{Grammatica}

Nu volgt een formele definitie van de syntaxis van de taal, aan de hand van een \BNF\ grammatica. Nummers zijn als volgt gedefiniëerd:
\begin{align*}
  \Num \GrammarDef (0 \mid 1 \mid 2 \mid 3 \mid 4 \mid 5 \mid 6 \mid 7 \mid 8 \mid 9)^+
\end{align*}
Eigenlijk gebruiken we geen strikte \BNF, in deze specifieke gevallen, maar een hele simpele variant, zoals \textsc{e-bnf}, waarbij de ook simpele reguliere expressies toelaten.% T:, die ook reguliere expressies toelaat.
Bovenstaand voorbeeld maakt dit duidelijk. Voorbeelden van elementen uit $\Num$ zijn ``0'', ``1'', ``235783'' en ``0003''. Voorbeelden van elementen die niet in $\Num$ zitten zijn ``'', ``-6'', ``4.2''.

\emph{Identifiers}, die gebruikt worden als variabel-namen,% T: namen voor variabelen en attributen
zijn op eenzelfde manier als volgt gedefiniëerd:
\begin{align*}
  \Id \GrammarDef (a \mid b \mid c \mid \dots \mid A \mid B \mid C \mid \dots)^+
\end{align*}
Hierbij moet je je natuurlijk% T: moet men zich
voorstellen dat alle letters uit het alfabet in de grammaticaregel staan op de voordehandliggende% T: voor de hand liggende
manier.

Het is soms ook nodig om meerdere komma-gescheiden namen te gebruiken, of een mogelijk lege lijst, zoals bij functie definities. Vandaar de volgende twee productieregels:
\begin{align*}
  \Ids \GrammarDef \Id \mid \Ids, \Id \\
  \MaybeIds \GrammarDef \varepsilon \mid \Ids
\end{align*}
Een \emph{slot}% T: pad, ook binnen de grammatica veranderen
is een opeenvolging door punten (``.'') gescheiden identifiers,% T: opeenvolging van identifiers gescheiden door punten
en wordt gebruikt om ook naar attributen van objecten te kunnen referenceren:% T: refereren
\begin{align*}
  \Path \GrammarDef \Id \mid \Id . \Path
\end{align*}
\emph{Expressies}, die ofwel primitieve waarden (getallen en functies), ofwel objecten kunnen weergeven, en \emph{boolse expressies}, die gebruikt worden voor loops en conditionele executie, zijn als volgt gedefiniëerd:% T: definiëren we als volgt
\begin{align*}
  \Expr \GrammarDef \Num \mid \Path \mid \varnothing \mid \Expr\; (+ \mid - \mid \times \mid /\,)\; \Expr \\
  \GrammarOr \FUN\pmb{(}\,\MaybeIds\,\pmb{)}\; \GrammarOpt{\RETURNS \Id\;} \pmb{\{}\; \Stm\; \pmb{\}} \\
  \Exprs \GrammarDef \Expr \mid \Exprs, \Expr \\
  \MaybeExprs \GrammarDef \varepsilon \mid \Exprs \\
  \BExpr \GrammarDef \TRUE \mid \FALSE \\
  \GrammarOr \BExpr\; (\AND \mid \OR)\; \BExpr \\
  \GrammarOr \NOT\; \BExpr \\
  \GrammarOr \Expr\; (=\: \mid\: <\: \mid\: \le\: \mid\: >\: \mid\: \ge)\; \Expr
\end{align*}
De kern van de hele grammatica draait om de volgende productieregels, die van \emph{statements}. Een statement is een programma van goede vorm. Het betekent niet nodelijkerwijs% T: per se / noodzakelijk
dat het programma \emph{valide} is, maar alle valide programma's zitten wel in $\Stm$.
\begin{align*}
  \Stm \GrammarDef \SKIP \\
  \GrammarOr \Stm; \Stm \\
  \GrammarOr \IF \BExpr \THEN \Stm \ELSE \Stm \\
  \GrammarOr \WHILE \BExpr \DO \Stm \\
  \GrammarOr \LOCAL \Id \\
  \GrammarOr \Id \OBJECT \\
  \GrammarOr \Id \CLONES \Id \\
  \GrammarOr \Path = \Expr \\
  \GrammarOr \GrammarOpt{\Path =} \Id\,\pmb{(}\,\MaybeExprs\,\pmb{)} \\
\end{align*}
Vanwege de focus van dit werkstuk definiëren we niet precies wanneer een programma valide is en wanneer niet.
% T: Nog iets over dat we vaak \Id\ gebruiken in plaats van \Path om het ons makkelijker te maken.
% T: Is er een reden dat we "Statements" afkorten tot "Stm", "Identifiers" tot "Id", "Numbers" tot "Num" en "Expression" niet tot "Expr"? Allemaal voluit lijkt mij het handigste, dan zijn ze ook goed te gebruiken in de tekst.

% vim: syn=latex spell spl=nl cole=1 cocu=nv
