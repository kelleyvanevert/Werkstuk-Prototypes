
\chapter{Taal en syntaxis}

In dit hoofdstuk zullen we de taal presenteren waarvoor we een natuurlijk taal construeren. De taal maakt gebruikt van prototype overerving en lexicaal bereik. Eerst zullen we een aantal voorbeeldprogramma's beschouwen, om zo informeel het karakter van de te formaliseren taal over te brengen. Daarna geven we een rigoreuze definitie met behulp van een BNF grammatica. De structuur van de productieregels van grammatica worden in latere hoofdstukken gebruikt om axioma's en deductieregels op te stellen. Daarmee heeft de grammatica in zekere zin een dubbele functie.

Elk voorbeeldprogramma en zijn toelichtingen worden als volg gepresenteerd:

	\begin{CodeFragment}[Het Eerste Voorbeeld]\label{hev}
		\Line{\VAR f}                      {variabelen moeten worden gedeclareerd}
		\Line{f = \FUN(i) \RETURNS n}      {}
		\Line{\IN \VAR n}                  {}
		\Line{\IN n = 2 \times (i + 5)}    {}
		\Line{}             {$x$ bestaat niet in deze scope}
		\Line{\VAR x}                      {$x$ is ongedefinieerd (maar wel aanwezig)}
		\Line{x = f(42)}                   {$x = 89$}
	\end{CodeFragment}

Bovenstaand code fragment is code fragment \ref{hev}.

De toelichtingen moeten als informeel commentaar worden beschouwd, waarmee we aan proberen te geven hoe het programma zich gedraagt. Vaak zijn het uitspraken over de toestand waarin het programma zich bevindt, direct na de linker regel te hebben ``uitgevoerd''.

\section{Voorbeeldprogramma's}

Een variabele moet gedeclareerd worden, en pas daarna kan er een waarde aan worden toegekend.

	\begin{CodeFragment}
		\Line{}                            {$x$ bestaat niet (in deze scope)}
		\Line{\VAR x}                      {$x$ is ongedefinieerd (maar wel aanwezig)}
		\Line{x = 5}                       {$x = 5$}
	\end{CodeFragment}

Het concept van declaratie is juist in deze taal, gezien het lexicaal bereik van variabelen, heel belangrijk. Vergelijk het bovenstaande programma fragment bijvoorbeeld met de volgende situatie.

Variabelen hebben geen vaste type. Er zijn drie typen waarden in de taal: getallen, functies en objecten.

	\begin{CodeFragment}
		\Line{\VAR x}                      {}
		\Line{x = 5}                       {de waarde van $x$ is een getal}
		\Line{x = \FUN()\,\{\;\SKIP\;\}}   {de waarde van $x$ is een functie}
		\Line{x \OBJECT}                   {de waarde van x is een object}
	\end{CodeFragment}

De taal is object georienteerd.

	\begin{CodeFragment}
		\Line{\VAR o}                      {}
		\Line{o \OBJECT}                   {}
		\Line{}                            {o.f is niet gedefinieerd}
		\Line{o.f = \FUN()\,\{\;\SKIP\;\}} {toekenning waarde aan object attribuut}
		\Line{}                            {o.f is wel gedefinieerd}
		\Line{o.n = 5}                     {}
	\end{CodeFragment}

Van de drie typen, zijn getallen en functies \emph{primitief}, en objecten \emph{niet primitief}. Primitieve waarde worden zelf gekopieerd (\emph{by-value}), maar van niet-primitieve waarden worden \emph{referenties} gekopieerd (\emph{by-reference}).

	\begin{CodeFragment}
		\Line{\VAR x;\; x = 6}             {}
		\Line{\VAR y;\; y = x}             {$x = 6$ en $y = 6$}
		\Line{y = 7}                       {$x = 6$ en $y = 7$}
		\Line{}                            {}
		\Line{\VAR p;\; p.n = 6}           {}
		\Line{\VAR q;\; q = p}             {$p$ en $q$ verwijzen nu naar hetzelfde object}
		\Line{}                            {$p.n = 6$ en $q.n = 6$}
		\Line{q.n = 7}                     {$p.n = 7$ en $q.n = 7$}
	\end{CodeFragment}

\subsection{Lexical scope}

Als in een zekere scope een variabele wordt gereferenceerd (nog) niet is gedefinieerd, wordt in omliggende scopes ``gezocht'' naar een definitie van deze variabele.

  \begin{CodeFragment}\label{exa:lexical}
		\Line{\VAR x}                      {}
		\Line{\VAR f;\; f = \FUN(i)}       {}
		\Line{\IN x = i + 5}               {}
		\Line{}                            {}
		\Line{f(5)}                        {$x = 10$}
	\end{CodeFragment}

..maar wanneer deze wel in de huidige scope bestaat, worden omliggende scopes ``met rust gelaten''.

	\begin{CodeFragment}
		\Line{\VAR x}                      {}
		\Line{\VAR f}                      {}
		\Line{f = \FUN(i)}                 {}
		\Line{\IN \VAR x}                  {}
		\Line{\IN x = i + 5}               {}
		\Line{}                            {}
		\Line{f(5)}                        {$x$ heeft nog geen waarde}
	\end{CodeFragment}

Telkens wanneer een functie wordt aangeroepen, wordt een \emph{nieuwe scope} aangemaakt voor lokale variabelen. Variabelen van deze nieuwe scope kunnen later nog gereferenceerd worden, doordat bijvoorbeeld de functie een lokale functie teruggeeft.

	\begin{CodeFragment}
		\Line{\VAR f}                      {}
		\Line{f = \FUN(n) \RETURNS g}      {}
		\Line{\IN \VAR g}                  {}
		\Line{\IN g = \FUN() \RETURNS n}   {}
		\Line{\IN \IN n = n + 1}           {}
		\Line{}                            {}
		\Line{\VAR c}                      {}
		\Line{c = f(5)}                    {$c() \rightarrow 6, 7, 8, \dots$}
	\end{CodeFragment}

\begin{lstlisting}[caption=Een countervoorbeeld,label=exa:counter]
local f
f = function(n) returns g
    local g
    g = function() returns n
        n = n + 1

local c
c = f(5)
c()                          # $\rightarrow 6, 7, 8, \dots$
\end{lstlisting}

\fbox{maar dan wat beter geschreven, etc...}

\subsection{Prototype overerving}

Prototype overerving is een variant van object-geörienteerd programmeren. De kern van object-geörienteerd programmeren is het concept van een \emph{object}, dat ertoe dient een verschijnsel uit de werkelijkheid na te bootsen (een reëel object, een patroon, een abstract idee). Het doel is om meer te kunnen programmeren op een conceptueel niveau. Daarmee wordt bijvoorbeeld zowel creatie als onderhoud van de code makkelijker.

Veel objecten zullen natuurlijk gelijke eigenschappen vertonen, of dezelfde structuur hebben. Verder wilt men concepten als specificering en generalisering toepassen op objecten. Deze problemen kunnen op meerdere manieren worden aangepakt. De bekendste variant is \emph{klasse gebaseerde} object-oriëntatie (ook wel \emph{klassieke object-oriëntatie}) en richt zich op het concept van een \emph{klasse}. Objecten van een bepaalde klasse vertonen de structuur en gedrag van die klasse en heten \emph{instanties}. Van specificering is sprake als een klasse eigenschappen van een andere klasse \emph{overerft}. Klassieke object-oriëntatie vind men in talen als Java en C\#.

Een andere aanpak met hetzelfde doel is \emph{prototype gebaseerde} object-oriëntatie. Daarbij wordt geen scheiding gemaakt tussen de concepten klasse, die structuur en gedrag specificeert, en instantie, die enkel deze eigenschappen vertoont. In plaats daarvan wordt gewerkt met een prototype structuur, waarbij elk object naar een bepaald \emph{prototype}-object refereert. Nu zijn objecten zelf de dragers van structuur en gedrag.
%Deze methode kan als flexibeler worden gezien, maar ook als een wat minder reëel beeld van de werkelijkheid worden opgevat.

Technisch gezien werkt prototype overerving als volgt. Van elk object is een prototype bekend, of het heeft geen prototype. Wanneer men een attribuut opvraagt van een zeker object, kan de op te leveren waarde procedureel als volgt worden opgevat:

\begin{enumerate}
  \item Bekijk of het attribuut gedefiniëerd is in het object zelf. In dat geval weten we de waarde en leveren deze op.
  \item Anders zoeken we het attribuut op in het prototype van het object. Ook dan weten we de waarde en leveren deze op.
  \item Wanneer ook het prototype het attribuut niet bevat, herhalen we de zoektocht voor alle volgende prototypen totdat we het attribuut hebben gevonden.
\end{enumerate}

Het grote verschil tussen object-gebaseerde talen en prototype-gebaseerde talen is dus dat de tweede geen onderscheid maakt tussen klassen en instanties. Een prototype heeft beide functies. Neem bijvoorbeeld het object @Deur@:
\begin{lstlisting}[name=deuren]
local Deur
Deur object
\end{lstlisting}
We kunnen @Deur@ direct als instantie gebruiken door een attribuut te zetten:
\begin{lstlisting}[name=deuren]
Deur.open = 1
\end{lstlisting}
Een @Deur@ is standaard open. We kunnen @Deur@ ook als een prototype gebruiken. In prototype-gebaseerde talen heet dit \emph{klonen}:
\begin{lstlisting}[name=deuren]
local GeslotenDeur
GeslotenDeur object
GeslotenDeur clones Deur
\end{lstlisting}
@GeslotenDeur@ heeft dan alle attributen van @Deur@:
\begin{lstlisting}[name=deuren]
GeslotenDeur.open            # => 1
\end{lstlisting}
Maar een @GeslotenDeur@ moet natuurlijk gesloten zijn. We zetten zijn attribuut @open@ op @0@:
\begin{lstlisting}[name=deuren]
GeslotenDeur.open = 0
\end{lstlisting}
Een gewone @Deur@ is nog steeds open:
\begin{lstlisting}[name=deuren]
Deur.open                    # => 1
\end{lstlisting}
Attributen worden dus per object bewaard. Door @open@ op @0@ te zetten in @GeslotenDeur@ verandert er niks in @Deur@.

We kunnen net zoveel klonen maken van een object als we willen en net zo diep klonen als we willen. Neem een @GlazenDeur@, dit is natuurlijk ook een @Deur@, maar wel doorzichtig:
\begin{lstlisting}[name=deuren]
local GlazenDeur
GlazenDeur object
GlazenDeur clones Deur
GlazenDeur.doorzichtig = 1
\end{lstlisting}
Een gewone @Deur@ heeft het attribuut @doorzichtig@ niet, en dus een @GeslotenDeur@ ook niet:
\begin{lstlisting}[name=deuren]
GeslotenDeur.doorzichtig     # => fout!
\end{lstlisting}
Maar we kunnen besluiten dat deuren standaard niet doorzichtig zijn:
\begin{lstlisting}[name=deuren]
Deur.doorzichtig = 0
\end{lstlisting}
Zodat ook onze @GeslotenDeur@ niet doorzichtig is:
\begin{lstlisting}[name=deuren]
GeslotenDeur.doorzichtig     # => 0
\end{lstlisting}
Maar er geld nog steeds:
\begin{lstlisting}[name=deuren]
GlazenDeur.doorzichtig       # => 1
\end{lstlisting}

We zien dat we met prototypes een zeer flexibele methode hebben om object-geörienteerd te programmeren. Het is niet nodig om de compiler of parser van te voren uit te leggen dat objecten aan bepaalde \enquote{blauwdrukken} moeten voldoen. We creëren objecten \enquote{on-the-fly}, alsmede hun attributen en relaties. Deze methode komt terug in talen als JavaScript, IO en Self.

Natuurlijk is het ook mogelijk om \emph{methoden} te definiëren. Dit zijn functie attributen gekoppeld aan een specifiek object. Stel dat we een @GeslotenDeur@ graag open willen maken. We definiëren:
\begin{lstlisting}[name=deuren]
GeslotenDeur.ontsluit = function (poging)
    if poging == this.code then
        this.open = 1
    else
        this.open = 0
\end{lstlisting}
@this@ is hier een expliciete verwijzing naar het huidige object. Op dit moment kunnen we @ontsluit@ nog niet aanroepen op @GeslotenDeur@:
\begin{lstlisting}[name=deuren]
GeslotenDeur.ontsluit(1234)  # => fout!
\end{lstlisting}
Het attribuut @code@ is immers niet gedefinieerd in @GeslotenDeur@ noch in zijn prototype @Deur@.

We kunnen natuurlijk een @code@ toekennen aan @GeslotenDeur@, maar laten we een specifieke @GeslotenDeur@ maken met een @code@:
\begin{lstlisting}[name=deuren]
local Kluis
Kluis object
Kluis clones GeslotenDeur
Kluis.code = 4321
\end{lstlisting}
Wanneer we de methode @ontsluit@ aanroepen is deze niet gedefinieerd in @Kluis@, maar wel in zijn prototype @GeslotenDeur@. Die wordt dan uitgevoerd. Een belangrijke observatie is dat @ontsluit@ wel wordt aangeroepen op @Kluis@. Dat betekent dat @this@ verwijst naar @Kluis@ en niet @GeslotenDeur@. Het attribuut @code@ wordt dan wel gevonden:
\begin{lstlisting}[name=deuren]
Kluis.ontsluit(1234)
Kluis.open                   # => 0
\end{lstlisting}
Helaas was dat de verkeerde code, we proberen het nog een keer:
\begin{lstlisting}[name=deuren]
Kluis.ontsluit(4321)
Kluis.open                   # => 1
\end{lstlisting}

\section{Grammatica}

Nu volgt een formele definitie van de syntaxis van de taal, aan de hand van een \BNF\ grammatica. Nummers zijn als volgt gedefiniëerd:
\begin{align*}
  \Num \GrammarDef (0 \mid 1 \mid 2 \mid 3 \mid 4 \mid 5 \mid 6 \mid 7 \mid 8 \mid 9)^+
\end{align*}
Eigenlijk gebruiken we geen strikte \BNF, in deze specifieke gevallen, maar een hele simpele variant, zoals \textsc{E-BNF}, waarbij de ook simpele reguliere expressies toelaten.%, die ook reguliere expressies toelaat.
Bovenstaand voorbeeld maakt dit duidelijk. Voorbeelden van elementen uit $\Num$ zijn ``0'', ``1'', ``235783'' en ``0003''. Voorbeelden van elementen die niet in $\Num$ zitten zijn ``'', ``-6'', ``4.2''.

\emph{Identifiers}, die gebruikt worden als variabel-namen,% namen voor variabelen en attributen
zijn op eenzelfde manier als volgt gedefiniëerd:
\begin{align*}
  \Id \GrammarDef (a \mid b \mid c \mid d \mid \dots)^+
\end{align*}
Hierbij moet je je natuurlijk% moet men zich
voorstellen dat alle letters uit het alfabet in de grammaticaregel staan op de voordehandliggende% voor de hand liggende
manier.

Het is soms ook nodig om meerdere komma-gescheiden namen te gebruiken, of een mogelijk lege lijst, zoals bij functie definities. Vandaar de volgende twee productieregels:
\begin{align*}
  \Ids \GrammarDef \Id \mid \Ids, \Id \\
  \MaybeIds \GrammarDef \varepsilon \mid \Ids
\end{align*}
Een \emph{slot}% pad, ook binnen de grammatica veranderen
is een opeenvolging door punten (``.'') gescheiden identifiers,% opeenvolging van identifiers gescheiden door punten
en wordt gebruikt om ook naar attributen van objecten te kunnen referenceren:% refereren
\begin{align*}
  \Slot \GrammarDef \Id \mid \Id . \Slot
\end{align*}
\emph{Expressies}, die ofwel primitieve waarden (getallen en functies), ofwel objecten kunnen weergeven, en \emph{boolse expressies}, die gebruikt worden voor loops en conditionele executie, zijn als volgt gedefiniëerd:% definiëren we als volgt
\begin{align*}
  \Expr \GrammarDef \Num \mid \Slot \mid \varnothing \mid \Expr\; (+ \mid - \mid \times \mid /\,)\; \Expr \\
  \GrammarOr \FUN\pmb{(}\,\MaybeIds\,\pmb{)}\; \GrammarOpt{\RETURNS \Id\;} \pmb{\{}\; \Stm\; \pmb{\}} \\
  \Exprs \GrammarDef \Expr \mid \Exprs, \Expr \\
  \ExprsMaybe \GrammarDef \varepsilon \mid \Exprs \\
  \B \GrammarDef \TRUE \mid \FALSE \\
  \GrammarOr \B\; (\AND \mid \OR)\; \B \\
  \GrammarOr \NOT\; \B \\
  \GrammarOr \Expr\; (=\: \mid\: <\: \mid\: \le\: \mid\: >\: \mid\: \ge)\; \Expr
\end{align*}
De kern van de hele grammatica draait om de volgende productieregels, die van \emph{statements}. Een statement is een programma van goede vorm. Het betekent niet nodelijkerwijs% per se / noodzakelijk
dat het programma \emph{valide} is, maar alle valide programma's zitten wel in $\Stm$.
\begin{align*}
  \Stm \GrammarDef \Stm; \Stm \\
  \GrammarOr \SKIP \\
  \GrammarOr \IF \B \THEN \Stm \ELSE \Stm \\
  \GrammarOr \WHILE \B \DO \Stm \\
  \GrammarOr \LOCAL \Id \\
  \GrammarOr \Id \OBJECT
  \GrammarOr \Id \CLONES \Id \\
  \GrammarOr \Slot = \Expr \\
  \GrammarOr \GrammarOpt{\Slot =} \Id\,\pmb{(}\,\ExprsMaybe\,\pmb{)} \\
\end{align*}
Vanwege de focus van dit werkstuk definiëren we niet precies wanneer een programma valide is en wanneer niet.

% vim: spell spl=nl
