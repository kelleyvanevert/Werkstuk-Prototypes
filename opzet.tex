\documentclass[a4paper,11pt]{article}
\usepackage[utf8]{inputenc}
\usepackage[T1]{fontenc}
\usepackage[dutch]{babel}

\usepackage{amsmath}
\usepackage{pxfonts}

\usepackage[margin=2cm]{geometry}
\usepackage{parskip}
\usepackage{lastpage}

\frenchspacing

\def\Title{Een Natuurlijke Semantiek voor JavaScript}
\def\Tim{Tim Steenvoorden (s0712663)}
\def\Kelley{Kelley van Evert (s4046854)}

\title{\Title \\ {\Large [\textit{Opzet werkstuk}]}}
\author{\Kelley \\ \Tim}
\date{maart 2012}

\usepackage{fancyhdr}
\pagestyle{fancy}
\lhead{\scriptsize \Title}
\chead{}
\rhead{\scriptsize [\textit{Opzet werkstuk}]\qquad\today}
\lfoot{\scriptsize \Kelley, \Tim}
\cfoot{}
\rfoot{\scriptsize \thepage/\pageref{LastPage}}
\renewcommand{\headrulewidth}{0pt}
\renewcommand{\footrulewidth}{0pt}

\begin{document}

\maketitle
\thispagestyle{fancy}

\section*{Onderwerpkeuze, vraagstelling en verwachtingen}

Het onderwerp van ons werkstuk is een specifieke variant van Object Oriëntatie. Deze variant is bekend van JavaScript, en kenmerkt zich door de volgende twee eigenschappen:

\begin{itemize}
	\item Lexicaal bereik
	\item Prototype overerving
\end{itemize}

Wij willen een natuurlijke semantiek ontwikkelen voor een minimale object geörienteerde taal die de bovenstaande twee taal constructies ondersteunt. De te ontwerpen taal zal erg op een kleine deelverzameling van JavaScript lijken.

\textbf{Wij hopen inzicht te krijgen in de manier waarop zulke taal constructies impact hebben op de natuurlijke semantiek van een object geörienteerde taal.} Daarbij kunnen wij ook een natuurlijke semantiek ontwikkelen die deze aspecten niet, of anders, ondersteunt. Denk hierbij bijvoorbeeld aan een minimale versie van Java, met klassieke overerving (\emph{inheritance}) in plaats van prototype overerving. Dit is een goede kandidaat voor een ``case study'' zoals aangegeven in de eisen voor het werkstuk.
%Onze bevindingen, bij het ontwerpen van deze natuurlijke semantieken, zullen als resultaat dienen van ons onderzoek.

De natuurlijke semantiek die wij willen ontwikkelen zal niet veel afwijken van die van While. De voornaamste verschillen zullen zich in het \emph{semantische model} voordoen, en daarmee dus ook de manier waarop de toestand van een programma zich laat representeren. Concrete veranderingen zijn bijvoorbeeld dat

\begin{itemize}
	\item het ``terminatie'' predikaat ($\longrightarrow$) een andere signatuur zal krijgen;
	\item er een aantal geheel nieuwe deductie regels bij zullen komen;
	\item de expressie interpretatie functie $\mathcal{A}$ aangepast en uitgebreid zal worden.
\end{itemize}

Maar los van deze aanpassingen zal de semantiek in zijn geheel niet fundamenteel veranderen. Terminatie uitspraken blijven van de vorm $\langle \mathit{Statement}, \mathit{state}\rangle \longrightarrow \mathit{state}'$, expressie interpretatie van de vorm $\mathcal{A}{[\![a]\!]}\mathit{state} = \mathit{waarde}$ en de evaluatie van expressies en statements zal strikt gescheiden blijven, zoals in While.

Wij verwachten daarmee dus ook het meeste werk te hebben aan het bedenken van een geschikt semantisch model. De syntaxis zal niet zo'n probleem zijn, aangezien het (1) ter ondersteuning is aan bovengenoemde taal constructies en dus zal voortvloeien uit de eisen die zich voordoen, en (2) waarschijnlijk niet veel hoeft af te wijken van de syntaxis van JavaScript, maar dan sterk vereenvoudigd.

\section*{Introductie lexicaal bereik}

\emph{Lexicaal bereik} (ook wel \emph{statisch bereik} genoemd) biedt grote expressiekracht aan programmeertalen zoals JavaScript. De manier waarop lexicaal bereik werkt lijkt heel erg op de manier waarop in wiskunde formules variabelen worden gebonden, en kan aan de hand van dit voorbeeld goed worden uitgelegd.

Beschouw de volgende (betekenisloze) wiskundige formule:

\begin{equation*}
	\sum_{x=1}^5{\sqrt{\frac{2x}{n}}}
\end{equation*}

Hierin is de variabele $x$ gebonden, en $n$ vrij. De gehele formule bevat vrije variabelen en is dus een open formule. Stel dat wij de gehele formule omvatten met een kwantor, waarmee het een uitspraak wordt:

\begin{equation*}
	\exists_n\; \mathrm{zodanig~dat}\;
	\sum_{x=1}^5{\sqrt{\frac{2x}{n}}}
	\le n^2
\end{equation*}

Nu refereert de variabele $n$ naar de $n$ die bij de kwantor staat, en is gebonden geworden. Net zo kan in een taal met lexicaal bereik, een vrije variabele ``altijd nog'' gebonden worden door er iets ``omheen'' te zetten.

Bekijk bijvoorbeeld de volgende twee pseudo code fragmenten. Het linker fragment wordt in het rechter fragment omvat met extra code, waardoor de variabele \textit{editing} gebonden wordt.

\begin{minipage}[t]{.5\textwidth}
	\begin{tabular}{rl}
		\small{1} & \textbf{var} $\ell = 5,$ \\
		\small{2} \\
		\small{3} & \phantom{\textbf{var}} handler =
									\textbf{function} $(e)\; \{$ \\
		\small{4} & \hspace{30pt} \textbf{if} $(e.\mathrm{code} == 27)$ \\
		\small{5} & \hspace{45pt} editing = \textsc{False}; \\
		\small{6} & \phantom{\textbf{var}} $\};$ \\
	\end{tabular}
\end{minipage}
\begin{minipage}[t]{.5\textwidth}
	\begin{tabular}{rl}
		\small{1} & \textbf{var} $\ell = 5,$ \\
		\small{2} & \phantom{\textbf{var}} editing = \textsc{True}, \\
		\small{3} & \phantom{\textbf{var}} handler =
									\textbf{function} $(e)\; \{$ \\
		\small{4} & \hspace{30pt} \textbf{if} $(e.\mathrm{code} == 27)$ \\
		\small{5} & \hspace{45pt} editing = \textsc{False}; \\
		\small{6} & \phantom{\textbf{var}} $\};$ \\
	\end{tabular}
\end{minipage}

Lexicaal bereik is vooral belangrijk bij het aanroepen van functies die variabelen daarbuiten aanpassen. Dit kunnen we illustreren met het volgende voorbeeld:

\begin{tabular}{rl}
    \small{1} & \textbf{var} $x = 1$ \\
    \small{2} & \phantom{\textbf{var}} $g =$ \textbf{function} $()\; \{$ \\
    \small{3} & \phantom{\textbf{var} $g =$ } \textbf{print} $x$; \\
    \small{4} & \phantom{\textbf{var} $g =$ } $x = 2$; \\
    \small{5} & \phantom{\textbf{var}} $\}$, \\
    \small{6} & \phantom{\textbf{var}} $f =$ \textbf{function} $()\; \{$ \\
    \small{7} & \phantom{\textbf{var} $f =$ } \textbf{var} $x = 3$; \\
    \small{8} & \phantom{\textbf{var} $f =$ } $g()$; \\
    \small{9} & \phantom{\textbf{var}} $\}$; \\
    \small{10} &  $f()$; \\
    \small{11} &  \textbf{print} $x$; \\
\end{tabular}

In de functie $f$ maken wij een nieuwe, lokale variabele $x$ aan die de globale variabele ``overschaduwd''. Wat gebeurt er nu wanneer we $f$ aanroepen in regel 10? In het geval van lexicaal bereik wordt $g$ uitgevoerd in de omgeving waarin deze gedefinieerd is, dus de omgeving met $x=1$. De uitvoer is dan ook $1, 2$. In het geval van dynamisch bereik zal $g$ de laatst gedeclareerde variabele $x$ gebruiken, dus $x=3$. De uitvoer zal dan $3, 1$ zijn.

\section*{Introductie prototype overerving}

Prototype overerving verschilt van klassieke overerving door geen onderscheid te maken tussen klassen en instanties. In plaats daarvan kunnen nieuwe objecten, door aan te geven welk ander object hun prototype is, structuur en functionaliteit overerven van willekeurige andere objecten. Deze manier van programmeren is ``losser'', en ``dynamischer'' van aard vergeleken met de strikte scheiding tussen definitie (\emph{class}) en gebruik (\emph{instance}) die van toepassing is bij klassieke overerving zoals bijvoorbeeld in Java, en daarmee ook een stuk expressiever.

De typische manier waarop een nieuw object wordt gemaakt met een ander object als prototype, is door te \textit{klonen}. Dat ziet er bijvoorbeeld als volgt uit:

\begin{tabular}{rl}
	\small{1} & $y$ = \textbf{clone} bestaand object $x$; \\
	\small{2} & voeg nieuwe functionaliteit, bijvoorbeeld: \\
	\small{3} & \quad \textbf{methode} a()$\dots$ \\
	\small{4} & \quad \textbf{variabele} b$\dots$ \\
	\small{5} & \quad et cetera$\dots$
\end{tabular}

JavaScript implementeert prototype overerving net iets anders, hoewel bovenstaande methode wel gesimuleerd kan worden. Elke functie ($F$) in JavaScript wordt gezien als een potentiële \textit{constructor}. Met de statement $y$ = \textbf{new} $F()$, wordt een nieuw object $y$ gemaakt, met als prototype het object dat opgeslagen was in $F$.prototype op het moment van evalueren van het statement, en vervolgens wordt $F$ direct uitgevoerd als constructor van dit nieuwe object.

Een veelgebruikte methode om de ``kloon-methode'' te simuleren in JavaScript is als volgt:

\begin{tabular}{rl}
	\small{1} & Object.create = \textbf{function}(proto) \{ \\
	\small{2} & $\quad$ \textbf{var} F = \textbf{function}() \{\}; \\
	\small{3} & $\quad$ F.prototype = proto; \\
	\small{4} & $\quad$ \textbf{return new} F(); \\
	\small{5} & \};
\end{tabular}

\end{document}

% vim: spell spl=nl ts=4
