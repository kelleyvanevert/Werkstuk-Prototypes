
\chapter{Semantisch model}

\begin{align*}
  \tag*{locaties van scopes en objecten}
  \sL &\DEF \{ (n, n) \in \sN^2 \} \\
  \tag*{functies}
  \sF &\DEF \Stm \times \Id_{\langle\rangle} \times (\Id \cup \{\BOT\}) \times \sL \\
  \tag*{waarden}
  \sV &\DEF \sL \cup \sN \cup \sF \\
  \tag*{binding-verzamelingen}
  \sB &\DEF \FiniteFunctions{\sV}{\Id} \\
  \tag*{objecten}
  \sO &\DEF \sB \times (\sL \cup \{\BOT\}) \\
  \tag*{scopes}
  \sS &\DEF \sB \times (\sL \cup \{\BOT\}) \\
\end{align*}

[Stukje bij beetje het semantisch model opbouwen, terwijl we steeds redeneren waarom we dat zo doen..]

\section{Scopes en lexical scope}

[Hierarchieen, outer scopes, bindingen, ``waarden'' kort noemen maar uitstellen tot ``Waarden: referenties en primitieven'']

\section{Objecten en prototype overerving}

[Graaf, bindingen, prototypen]

\section{Waarden: referenties en primitieven}

[Ze worden op dezelfde manier behandeld: objecten by-reference, dus de references zelf by-value, net als primitieven -- vandaar dat ze in dezelfde verzameling waarden zitten.]